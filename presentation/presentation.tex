% Created 2013-07-25 Thu 18:08
\documentclass[bigger]{beamer}
\usepackage[utf8]{inputenc}
\usepackage{tikz}
\usetikzlibrary{decorations,arrows,shapes}
\usepackage[T1]{fontenc}
\usepackage{fixltx2e}
\usepackage{graphicx}
\usepackage{longtable}
\usepackage{float}
\usepackage{wrapfig}
\usepackage{soul}
\usepackage{textcomp}
\usepackage{marvosym}
\usepackage{wasysym}
\usepackage{latexsym}
\usepackage{amssymb}
\usepackage{hyperref}
\usepackage{bm}
\usepackage{ulem}
\tolerance=1000
\usepackage{amsmath} \usepackage{algorithm} \usepackage{graphicx}
\usepackage{amssymb} \usepackage{mathrsfs} 
\newcommand{\bs}[1]{\bm{#1}}
\newcommand{\ie}{{\em i.e.},~}
\newcommand{\eg}{{\em e.g.},~}
\newcommand{\parentd}{../}
\newcommand{\R}{\mathbb{R}}
\newcommand{\indf}[1]{\mathbf{1}_{\left\{#1\right\}}}
\providecommand{\alert}[1]{\textbf{#1}}
\usetheme{Novartis}
\doNotUseNovartisLogoOnAllSlides
\doNotUseNovartisLogoOnTitlePage


\title{Matroids from families of null hypotheses}
\author{Florian Klinglmueller, Willi Maurer}
\date{\today}
\hypersetup{
  pdfkeywords={},
  pdfsubject={},
  pdfcreator={Emacs Org-mode version 7.8.11}}

\begin{document}
\frametitle{Matroids}

\maketitle



\begin{frame}
\frametitle{Outline}
\setcounter{tocdepth}{3}
\tableofcontents
\end{frame}




\section{Logically dependent hypotheses}


\begin{frame}
\frametitle{Pairwise comparisons, some history }

Testing hypotheses of no difference in pairwise comparisons of 3 treatments (categories):
\begin{itemize}
  \item Parameters: $\mu_1,\mu_2,\mu_3$ ; \\ Hypotheses: $H_1:\mu_1=\mu_2,H_2:\mu_2=\mu_3;H_3:\mu_1=\mu_3$
  \item A level-$\alpha$ test of $H_ {\{1,2,3\}} = H_1\cap H_2 \cap H_3 : \mu_1=\mu_2=\mu_3$. followed by level-$\alpha$ tests of $H_1,H_2,H_3$ protects the FWER at level $\alpha$.
  \begin{itemize}
    \item  essentially closed test; no need to test pairwise intersection hypotheses since $H_{{i,j}} = H_{\{1,2,3\}}, i\neq j$,
    \item  Bonferroni: if one of the hypotheses is rejected at level $\alpha/3$ the others can be tested at level alpha.
    \item  Hierarchical: if a predetermined primary hypothesis is rejected at level $\alpha$ the other two can be tested at level $\alpha$.
    \item  $\Rightarrow$ improves respective Bonferroni-Holm or fixed sequence tests.
  \end{itemize}
\end{itemize}
\end{frame}

\begin{frame}
\frametitle{Pairwise comparisons, some history}

Testing hypotheses of no difference in pairwise comparisons of $m>3$ treatments (categories):\\
\cite{shaffer1986modified}: Modified Sequentially Rejective Multiple test procedures; Example: Modified Bonferroni-Holm
\begin{itemize}
  \item "Static":  After j rejections, test remaining hypotheses at level $\alpha/t_j$, $t_j$ = max number of true hypotheses if $j$ are false.
  \item "Dynamic": After rejecting $H_{i_1},...,H_{i_j}$, test remaining hypotheses at level $\alpha/t^*_j$,  $t^*_j$ = number of true hypotheses if $H_{i_1},...,H_{i_j}$ are false.
  \item Example: $m=4, n=6$ :
    \begin{itemize}
      \item Static: $t_0=6, t_1,t_2,t_3 = 3, t_4=2, t_5=1$
      \item Dynamic: if, e.g., $H_{\{1,2\}}$ and $H_{\{3,4\}}$ are rejected then only $H_{\{1,3\}}$ and $H_{\{2,3\}}$, or $H_{\{1,4\}}$ and $H_{\{2,3\}}$ can be simultaneously true, hence $t^*_2 = 2 < t_2 =3$ in this case.
       \end{itemize}
\end{itemize}
\end{frame}




\begin{frame}
\frametitle{Pairwise mean comparisons (two-sided)}
\begin{itemize}

\item Consider $\mu_i, i \in \{1,...,m\}$

\item $H_k: \mu_i = \mu_j, k \in S = \{1,...,n\}$, where $n \leq m*(m-1)/2$

\item Can be represented by an undirected graph $\mathscr{G} = (V,E)$:
\begin{itemize}
\item vertices correspond to means, \emph{i.e.}, $V = \{\mu_1,...,\mu_m\}$,
\item edges correspond to hypotheses, \emph{i.e.}, $E =
      \left\{e_1,...,e_n\} \right\}$, $e_l = \{\mu_i,\mu_j\}$, $i \neq
      j$, and $e_l \neq e_k$ for $l \neq k$
\end{itemize}
\item Let $H_I = \bigcap_{k \in I} H_k$, $I \subseteq S$
\item $I$ induces a subgraph $\mathscr{G}_I$ on the edge set $E_I=\{e_i: i
  \in I\}$
\end{itemize} % ends low level
\end{frame}

\begin{frame}
\frametitle{Example 1: 4 Groups}
\begin{tikzpicture}[scale=.5]

\node (H1) at (225bp,-125bp)[draw,circle,fill=white] {$\mu_1$};
\node (H2) at (425bp,-125bp)[draw,circle,fill=white] {$\mu_2$};
\node (H3) at (425bp,-325bp)[draw,circle,fill=white] {$\mu_3$};
\node (H4) at (225bp,-325bp)[draw,circle,fill=white] {$\mu_4$};
\draw [line width=1pt] (H1) to (325bp, -124bp) node[fill=blue!20] {$H_{1}$} to (H2);
\draw [line width=1pt] (H1) to (275bp, -175bp) node[fill=blue!20] {$H_{2}$} to (H3);
\draw [line width=1pt] (H2) to (425bp, -224bp) node[fill=blue!20] {$H_{4}$} to (H3);
\draw [line width=1pt] (H3) to (326bp, -324bp) node[fill=blue!20] {$H_{6}$} to (H4);
\draw [line width=1pt] (H4) to (225bp, -227bp) node[fill=blue!20] {$H_{3}$} to (H1);
\draw [line width=1pt] (H4) to (374bp, -176bp) node[fill=blue!20] {$H_{5}$} to (H2);


\node (H1) at (525bp,-125bp)[draw,circle,fill=white] {$\mu_1$};
\node (H2) at (725bp,-125bp)[draw,circle,fill=white] {$\mu_2$};
\node (H3) at (725bp,-325bp)[draw,circle,fill=white] {$\mu_3$};
\node (H4) at (525bp,-325bp)[draw,circle,fill=white] {$\mu_4$};
\draw [line width=1pt] (H1) to (625bp, -124bp) node[fill=red!20] {$H_{1}$} to (H2);
\draw [line width=1pt] (H1) to (575bp, -175bp) node[fill=blue!20] {$H_{2}$} to (H3);
\draw [line width=1pt] (H2) to (725bp, -224bp) node[fill=blue!20] {$H_{4}$} to (H3);
\draw [line width=1pt] (H3) to (626bp, -324bp) node[fill=red!20] {$H_{6}$} to (H4);
\draw [line width=1pt] (H4) to (525bp, -227bp) node[fill=blue!20] {$H_{3}$} to (H1);
\draw [line width=1pt] (H4) to (674bp, -176bp) node[fill=blue!20] {$H_{5}$} to (H2);


\end{tikzpicture}

\end{frame}



% \begin{frame}
% \frametitle{Example 1: 4 Means}
% \begin{tikzpicture}[scale=1]
\node (H1) at (225bp,-125bp)[draw,circle,fill=white] {$\mu_1$};
\node (H2) at (425bp,-125bp)[draw,circle,fill=white] {$\mu_2$};
\node (H3) at (425bp,-325bp)[draw,circle,fill=white] {$\mu_3$};
\node (H4) at (225bp,-325bp)[draw,circle,fill=white] {$\mu_4$};
\draw [line width=1pt] (H1) to (325bp, -124bp) node[fill=blue!20] {$H_{1}$} to (H2);
\draw [line width=1pt] (H1) to (275bp, -175bp) node[fill=blue!20] {$H_{2}$} to (H3);
\draw [line width=1pt] (H2) to (425bp, -224bp) node[fill=blue!20] {$H_{4}$} to (H3);
\draw [line width=1pt] (H3) to (326bp, -324bp) node[fill=blue!20] {$H_{6}$} to (H4);
\draw [line width=1pt] (H4) to (225bp, -227bp) node[fill=blue!20] {$H_{3}$} to (H1);
\draw [line width=1pt] (H4) to (374bp, -176bp) node[fill=blue!20] {$H_{5}$} to (H2);
\end{tikzpicture}

% \end{frame}



\begin{frame}
\frametitle{General contrast tests (two-sided)}
\begin{itemize}

\item Consider $m$-variate random variable $\bs{X} \sim F(\bs{\mu})$, $\bs{\mu} = (\mu_1,...,\mu_m)^t$

\item Contrasts $\bs{c}_i = (c_{i1},...,c_{im})^t$, $\sum_{j = 1}^m
  c_{ij} = 0$
\item Contrast matrix  $\bs{C} = \left( \bs{c}_1,...,\bs{c}_n\right)$,
  with sub-matrix $\bs{C}_I = \left( \bs{c}_i \right)_{i \in
    I}$
\item Elementary hypotheses $H_i: \bs{c}_i^t\bs{\mu} = 0$,
\item Intersection hypotheses $H_I: \bs{C}_I^t\bs{\mu} = 0$

\item Elementary hypotheses are represented by column vectors,
  intersection hypotheses by sub-matrices of the contrast matrix

\end{itemize} % ends low level
\end{frame}



\begin{frame}
\frametitle{Example 2: Contrast test}


\begin{tabular}{ll}
  \begin{minipage}{.45\textwidth}
    All pairwise comparisons of 4 means 
    \begin{displaymath}
    \bs{C}^t = \left(
      \begin{array}{llll}
      -1 & 1 & 0 & 0 \\
      -1 & 0 & 1 & 0 \\
      -1 & 0 & 0 & 1 \\
      0 & -1 & 1 & 0 \\
      0 & -1 & 0 & 1 \\
      0 & 0 & -1 & 1 
    \end{array}
    \right)
    \end{displaymath}
  \end{minipage} &
  \begin{minipage}{.55\textwidth}
    Modified Williams Contrasts for Dose Response
    \begin{displaymath}
      \bs{C}^t=\left(
      \begin{array}{llll}
        1 & 0 & 0 & -1\\
        1 & 0 &-.51 & -.49 \\
        1 & -.35 & -.33 & -.32\\
        .51 & .49 & -.51 & -.48\\
        .51 & .49 & 0 &  -1\\
        .35 & .33 & .32& -1
      \end{array}\right)
    \end{displaymath}
  \end{minipage}
\end{tabular}

\end{frame}



\begin{frame}
\frametitle{Dependent null hypotheses}

\begin{itemize}
\item $H_i$ depends on $H_I$ iff $H_{I \cup i} = H_{I}$, induces a
  dependence relation $\sim_H$ on $\mathcal{P}(S)$
\item $H_I$ independent iff for all $i \in I$, $i \nsim_H I$,
\item $I$ independent $\Leftrightarrow$ $H_{I} \subsetneq
  H_{I\setminus i}$ for all $i \in I$
\pause
\item For pairwise comparisons $I$ independent iff there are no cycles in $\mathscr{G}_I$
\item For general contrast tests $I$ independent iff $\bs{C_I}$ has full
  column rank
\end{itemize} % ends low level
\end{frame}



\section{Matroids}



\begin{frame}
\frametitle{Matroids}
\begin{itemize}

\item Matroids generalize the structures of graphs and vector spaces,
  \eg cycles, bases, independent sets
\item A matroid $M$ on $S$ endows $\mathcal{P}(S)$ with an algebraic
  structure 
\item Cryptomorphism: many ways to define a matroid: 
  \begin{itemize}
  \item From a graph perspective: Circuit axioms.
  \item From a vector space perspective: Independence axioms.
  \end{itemize}
\item There is a dual matroid $M^*$ which can be helpful for proofs and algorithms 
\end{itemize} % ends low level
\end{frame}



\begin{frame}
\frametitle{Circuit axioms}


\begin{tabular}{ll}
  \begin{minipage}{.4\textwidth}
    \only<2->{\begin{tikzpicture}[scale=.5]
\node (H1) at (225bp,-125bp)[draw,circle,fill=white] {$\mu_1$};
\node (H2) at (425bp,-125bp)[draw,circle,fill=white] {$\mu_2$};
\node (H3) at (425bp,-325bp)[draw,circle,fill=white] {$\mu_3$};
\node (H4) at (225bp,-325bp)[draw,circle,fill=white] {$\mu_4$};
\draw [line width=1pt] (H1) to (325bp, -124bp) node[fill=blue!20] {$H_{1}$} to (H2);
\draw [line width=1pt] (H1) to (275bp, -175bp) node[fill=blue!20] {$H_{2}$} to (H3);
\draw [line width=1pt] (H2) to (425bp, -224bp) node[fill=blue!20] {$H_{4}$} to (H3);
\end{tikzpicture}
}
.
  \end{minipage} &
  \begin{minipage}{.6\textwidth}
    Consider $S$ and a collection of subsets $\mathscr{C}$ for which holds
    \begin{enumerate}
    \item $C_1 \neq C_2 \in \mathscr{C}$, then $C_1 \nsubseteq C_2$,
    \item $C_1 \neq C_2 \in \mathscr{C}$, and $i \in C_1 \cap C_2$ then
      $\exists C_3 \subseteq (C_1 \cup C_2)$,
    \end{enumerate}
    then $\mathscr{C}$ is the set of circuits of a matroid $M$ on $S$
  \end{minipage}
\end{tabular}



\end{frame}

% \item \textbf{/Some illustration hereabouts/}

% \end{itemize} % ends low level
% \end{frame}


\begin{frame}
\frametitle{Independence axioms}

\begin{tabular}{ll}
  \begin{minipage}{.4\textwidth}
  \only<1>{\begin{tikzpicture}[scale=.5]
\node (H1) at (225bp,-125bp)[draw,circle,fill=white] {$\mu_1$};
\node (H2) at (425bp,-125bp)[draw,circle,fill=white] {$\mu_2$};
\node (H3) at (425bp,-325bp)[draw,circle,fill=white] {$\mu_3$};
\node (H4) at (225bp,-325bp)[draw,circle,fill=white] {$\mu_4$};
\draw [line width=1pt] (H1) to (325bp, -124bp) node[fill=blue!20] {$H_{1}$} to (H2);
\draw [line width=1pt] (H2) to (425bp, -224bp) node[fill=blue!20] {$H_{4}$} to (H3);
\end{tikzpicture}
}
  \only<2>{\begin{tikzpicture}[scale=.5]
\node (H1) at (225bp,-125bp)[draw,circle,fill=white] {$\mu_1$};
\node (H2) at (425bp,-125bp)[draw,circle,fill=white] {$\mu_2$};
\node (H3) at (425bp,-325bp)[draw,circle,fill=white] {$\mu_3$};
\node (H4) at (225bp,-325bp)[draw,circle,fill=white] {$\mu_4$};
\draw [line width=1pt] (H1) to (325bp, -124bp) node[fill=blue!20] {$H_{1}$} to (H2);
\draw [line width=1pt] (H1) to (275bp, -175bp) node[fill=blue!20] {$H_{2}$} to (H3);
\draw [line width=1pt] (H2) to (425bp, -224bp) node[fill=blue!20] {$H_{4}$} to (H3);
\end{tikzpicture}
}
  \invisible<1>{\alert{\footnotesize minimally dependent set!}}
  \end{minipage} &
  \begin{minipage}{.6\textwidth}
  Consider $S$ and a collection of subsets $\mathscr{I}$ for which
  holds
  \begin{enumerate}
  \item $\emptyset \in \mathscr{I}$,
  \item $I \in \mathscr{I}$ and $J \subseteq I$ then $J \in
    \mathscr{I}$,
  \item $I,J \in \mathscr{I}$, $|I| = |J + 1|$ then $\exists j \in I
    \setminus J$ such that $J \cup i \in \mathscr{I}$,
  \end{enumerate}
  then $\mathscr{I}$ is the set of independent sets of a matroid $M$
  on $S$. This defines an independence relation $i \nsim_M I$ iff
  $I\cap i \in \mathscr{I}$. 

  \end{minipage}
\end{tabular}




\end{frame}


\begin{frame}
\frametitle{Dependence axioms}

\begin{tabular}{ll}
  \begin{minipage}{.4\textwidth}
    \begin{tikzpicture}[scale=.5]
\node (H1) at (225bp,-125bp)[draw,circle,fill=white] {$\mu_1$};
\node (H2) at (425bp,-125bp)[draw,circle,fill=white] {$\mu_2$};
\node (H3) at (425bp,-325bp)[draw,circle,fill=white] {$\mu_3$};
\node (H4) at (225bp,-325bp)[draw,circle,fill=white] {$\mu_4$};
\draw [line width=1pt] (H1) to (325bp, -124bp) node[fill=blue!20] {$H_{1}$} to (H2);
\draw [line width=1pt] (H1) to (275bp, -175bp) node[fill=blue!20] {$H_{2}$} to (H3);
\draw [line width=1pt] (H2) to (425bp, -224bp) node[fill=blue!20] {$H_{4}$} to (H3);
\draw [line width=1pt] (H4) to (225bp, -227bp) node[fill=blue!20] {$H_{3}$} to (H1);
\end{tikzpicture}

  \end{minipage} &
  \begin{minipage}{.6\textwidth}
  Relation $\sim_M$ on elements and subsets of $S$,
  if for any $x \in S$, and sets $Y,Z \subseteq S$ holds
  \begin{enumerate}
  \item $y \sim Y$ for all $y \in Y$
  \item if $x \sim_M Y$ and $x \nsim_M Y\setminus y$
    then $y \sim x \cup (Y \setminus y)$
  \item if $x \sim_M Y$ and for all $y \in Y$, $y \sim_M Z$ then $x \sim_M Z$
  \end{enumerate}
  then $\sim_M$ defines a matroid on $S$.
  \end{minipage}
\end{tabular}



\end{frame}

\section{Matroids from logically dependent hypotheses}
\begin{frame}
  \frametitle{Dependent hypotheses}
  \begin{block}{Equivalence of dependence relations $\sim_H = \sim_M$}
    \begin{itemize}
    \item<1-> $H_i$ depends on $H_I$ iff $H_{I \cup i} = H_{I}$
    \item<2-> Does this mean $\mathscr{H}$ defines a matroid $M$, such that
      $\sim_H = \sim_M$?
    \end{itemize}
  \end{block}
\end{frame}



\begin{frame}
\frametitle{Matroids from logically dependent hypotheses}

  \begin{block}{Two-sided pariwise comparisons}
    For a family of two-sided pairwise mean comparisons $H_i$, $i \in
    S$, an elementary hypotheses $H_j$ depends on intersection
    hypotheses $H_I$ iff the corresponding edge $e_j$ is dependent on
    the edge set $\{e_i:i \in I\}$ of the cycle matroid defined by 
    $\mathscr{G}_I$.   
  \end{block}

  \begin{block}{Two-sided linear contrast tests}
    For a family of two-sided general contrast tests $H_i$, $i \in S$,
    an elementary hypothesis $H_j$ depends on intersection hypotheses
    $H_I$ iff the corresponding contrast vector $\bs{c_j}$ is
    dependent on the column vectors of the matrix $\bs{C}_I$ in the
    vectorial matroid defined by $\bs{C}$ on $\mathbb{R}^m$.
  \end{block}

\end{frame}


\begin{frame}
  \frametitle{Matroids from null hypotheses}
  {\Large What is it good for?\\
  \invisible<1>{Structure}}
\end{frame}

\begin{frame}
\frametitle{Maximally independent sets}
  \begin{itemize}
  \item Bases are maximally independent sets of $S$
  \item Rank $\rho I$ is the cardinality of a maximally indepedendent set in $I$
  \item Circuits are minimally dependent sets, \ie each subset of $C$
    with rank $\rho C - 1$ is a maximally independent set of $C$
  \end{itemize}
\end{frame}



\begin{frame}
\frametitle{Maximally dependent sets}
\begin{itemize}
\item The closure of a set $\sigma I = \{i \in S: i \sim I\}$
\item Closed sets {\em flats}: $\mathscr{L} = \{\sigma I: I \subseteq S\}$
\item Hyperplanes are flats of rank $\rho S - 1$
\item All flats are given by closing hyperplanes under intersection
\item Circuits in $M^*$ are the complements of hyperplanes in $M$
\end{itemize} % ends low level
\end{frame}



\begin{frame}{Examples}
\framesubtitle<1>{Independent set}  
\framesubtitle<2>{Base}  
\framesubtitle<4>{Dependent set}  
\framesubtitle<3>{Cycle}  
\framesubtitle<5>{Closure}  
\only<1>{\begin{tikzpicture}[scale=.5]
\node (H1) at (225bp,-125bp)[draw,circle,fill=white] {$\mu_1$};
\node (H2) at (425bp,-125bp)[draw,circle,fill=white] {$\mu_2$};
\node (H3) at (425bp,-325bp)[draw,circle,fill=white] {$\mu_3$};
\node (H4) at (225bp,-325bp)[draw,circle,fill=white] {$\mu_4$};
\draw [line width=1pt] (H1) to (325bp, -124bp) node[fill=blue!20] {$H_{1}$} to (H2);
\draw [line width=1pt] (H2) to (425bp, -224bp) node[fill=blue!20] {$H_{4}$} to (H3);
\end{tikzpicture}
}
\only<2>{\begin{tikzpicture}[scale=.5]
\node (H1) at (225bp,-125bp)[draw,circle,fill=white] {$\mu_1$};
\node (H2) at (425bp,-125bp)[draw,circle,fill=white] {$\mu_2$};
\node (H3) at (425bp,-325bp)[draw,circle,fill=white] {$\mu_3$};
\node (H4) at (225bp,-325bp)[draw,circle,fill=white] {$\mu_4$};
\draw [line width=1pt] (H1) to (325bp, -124bp) node[fill=blue!20] {$H_{1}$} to (H2);
\draw [line width=1pt] (H2) to (425bp, -224bp) node[fill=blue!20] {$H_{4}$} to (H3);
\draw [line width=1pt] (H4) to (225bp, -227bp) node[fill=blue!20] {$H_{3}$} to (H1);
\end{tikzpicture}
}
\only<4>{\begin{tikzpicture}[scale=.5]
\node (H1) at (225bp,-125bp)[draw,circle,fill=white] {$\mu_1$};
\node (H2) at (425bp,-125bp)[draw,circle,fill=white] {$\mu_2$};
\node (H3) at (425bp,-325bp)[draw,circle,fill=white] {$\mu_3$};
\node (H4) at (225bp,-325bp)[draw,circle,fill=white] {$\mu_4$};
\draw [line width=1pt] (H1) to (325bp, -124bp) node[fill=blue!20] {$H_{1}$} to (H2);
\draw [line width=1pt] (H1) to (275bp, -175bp) node[fill=blue!20] {$H_{2}$} to (H3);
\draw [line width=1pt] (H2) to (425bp, -224bp) node[fill=blue!20] {$H_{4}$} to (H3);
\draw [line width=1pt] (H4) to (225bp, -227bp) node[fill=blue!20] {$H_{3}$} to (H1);
\end{tikzpicture}
}
\only<3>{\begin{tikzpicture}[scale=.5]
\node (H1) at (225bp,-125bp)[draw,circle,fill=white] {$\mu_1$};
\node (H2) at (425bp,-125bp)[draw,circle,fill=white] {$\mu_2$};
\node (H3) at (425bp,-325bp)[draw,circle,fill=white] {$\mu_3$};
\node (H4) at (225bp,-325bp)[draw,circle,fill=white] {$\mu_4$};
\draw [line width=1pt] (H1) to (325bp, -124bp) node[fill=blue!20] {$H_{1}$} to (H2);
\draw [line width=1pt] (H1) to (275bp, -175bp) node[fill=blue!20] {$H_{2}$} to (H3);
\draw [line width=1pt] (H2) to (425bp, -224bp) node[fill=blue!20] {$H_{4}$} to (H3);
\end{tikzpicture}
}
\only<5>{\begin{tikzpicture}[scale=.5]
\node (H1) at (225bp,-125bp)[draw,circle,fill=white] {$\mu_1$};
\node (H2) at (425bp,-125bp)[draw,circle,fill=white] {$\mu_2$};
\node (H3) at (425bp,-325bp)[draw,circle,fill=white] {$\mu_3$};
\node (H4) at (225bp,-325bp)[draw,circle,fill=white] {$\mu_4$};
\draw [line width=1pt] (H1) to (325bp, -124bp) node[fill=blue!20] {$H_{1}$} to (H2);
\draw [line width=1pt] (H1) to (275bp, -175bp) node[fill=blue!20] {$H_{2}$} to (H3);
\draw [line width=1pt] (H2) to (425bp, -224bp) node[fill=blue!20] {$H_{4}$} to (H3);
\draw [line width=1pt] (H3) to (326bp, -324bp) node[fill=blue!20] {$H_{6}$} to (H4);
\draw [line width=1pt] (H4) to (225bp, -227bp) node[fill=blue!20] {$H_{3}$} to (H1);
\draw [line width=1pt] (H4) to (374bp, -176bp) node[fill=blue!20] {$H_{5}$} to (H2);
\end{tikzpicture}
}
\end{frame}








\begin{frame}
\frametitle{Flats}
\begin{itemize}

\item The flats of a matroid are closed under intersection

\item Every elementary hypothesis is a flat
\item For each $I \subseteq S$, $H_I = H_{\sigma I}$
\item Define a stepwise procedure on the flats of $M$


\end{itemize} % ends low level
\end{frame}

\begin{frame}
  \frametitle{Stepwise test procedure}
  \begin{block}{Multiple testing procedure for logically related hypotheses}
    Assume that $\mathscr{H}$ with dependence relation $\sim_H$
    defines a matroid on $M$ on $S$. Then a test procedure that
    rejects elementary hypotheses $H_i$ if all $H_E$, $i \in E$, $E
    \in \mathscr{L}$ are rejected using a local level-$\alpha$ test
    controls the FWER in the strong sense.
  \end{block}

\end{frame}

\begin{frame}
  \frametitle{Stepwise multiple testing procedure on the lattice of
    flats}

\begin{tabular}{ll}
  \begin{minipage}{.3\textwidth}
    \includegraphics<1>[width=1.2\textwidth]{lattice4.pdf}
    \includegraphics<2>[width=1.2\textwidth]{lattice4accept.pdf}
    \includegraphics<3>[width=1.2\textwidth]{lattice4reject.pdf}
    \includegraphics<4>[width=1.2\textwidth]{lattice4rejacc.pdf}\\
    {\tiny wikipedia:Partition\_of\_a\_Set}
  \end{minipage} &
  \begin{minipage}{.7\textwidth}
  \begin{itemize}
  \item All flats are given by intersections of hyperplanes
    $\rightarrow$ knowing all hyperplanes remaining flats are given by
    intersection of index sets
  \item Hyperplanes can by found by algorithms that enumerate the
    circuits of the dual matroid \cite{boros2003algorithms}
  \item The partial order of flats defined by inclusion defines a
    geometric lattice
  \item<2-> Testing only a subset of elementary hypotheses corresponds
    to intervals
  \item<3-> Rejection of elementary hypotheses removes intervals
  \end{itemize}

  \end{minipage}
\end{tabular}

\end{frame}

\begin{frame}{Matroid procedures for pairwise mean comparisons}
\begin{tabular}{ll}
  \begin{minipage}{.3\textwidth}
    \includegraphics[width=1.1\textwidth]{setpartitions.pdf} \\
     {\tiny wikipedia:Bell\_number}
  \end{minipage} &
  \begin{minipage}{.7\textwidth}
  \begin{itemize}
  \item Two-sided comparisons of $m$ means
  \item The cycle matroid of the graph representation
    provides equivalent dependence structure
  \item The flats of a (simple) complete graph are
    isomporphic to the partitions of a set of size $m$
  \item Efficient algorithms to generate all exhaustive sets by
    enumeration of set partitions \cite{er1988fast,knuth2005art,kokosinski2006new} 
  \item The number of flats are $m$-th Bell-Number $-1$ \ie 4, 14,
    51, 202, 876, 4139, ... 
  \end{itemize}
  \end{minipage}
\end{tabular}
  
\end{frame}




\begin{frame}
\frametitle{Connections to existing procedures}

\textbf{Old wine in new bottles ((c) L. Hothorn)}
\begin{itemize}
\item \cite{shaffer1986modified} represents pairwise comparisons
  by set partitions 
\item \cite{bergmann1988improvements} define the concepts of an
  exhaustive sets
\item \cite{bernhard1991computergestuetzte} shows properties of
  exhaustive sets that are basically {\em closure axioms} of a matroid
  and provides algorithms to check each intersection hypotheses if is
  a flat using rank function.
\item \cite{westfall1997multiple} generates flats using dependence
  arguments in vector spaces.
\item \cite{weichert2000robuste} suggest an MTP based on clique
  decomposition of the graph, cliques in the graph are flats
\item \cite{westfall2007multiple} provide a branch and bound algorithm
  to generate all flats. They note that for pairwise comparisons all
  exhaustive hypotheses are {\em partition hypotheses} 
\end{itemize}
\end{frame}


\begin{frame}
\frametitle{Matroid representation}
  \begin{itemize}
  \item proofs are shorter (more elegant),
  \item efficient algorithms exist to find the flats of matroids,
  \item {\em Scum principle}\cite{welsh2010matroid} everything
    important happens at the top (hyperplanes)
  \item we may define procedures using independent sets,
  \item \ldots{}
  \end{itemize}
\end{frame}

\section{Graphical approaches for logically dependent hypotheses}



\begin{frame}
\frametitle{Reminder: Graphical approaches}

\begin{tabular}{ll}
  \begin{minipage}{.4\textwidth}
    \begin{tikzpicture}[remember picture,overlay]
      \node[yshift=-2.5cm,xshift=-1.4cm] at (current page.north west){\input{qvatest}};
    \end{tikzpicture}

  \end{minipage} &
  \begin{minipage}{.6\textwidth}
    \begin{itemize}
    \item Provides weights $w_{i,I}$ for each index $i$ and each
      non-empty subset $I \subset S$
    \item Weights are used to define a closed test procedure of
      weighted intersection hypotheses tests
    \item We have $\sum_{i \in S} w_{i,I} \leq 1$ (conservative),
    \item and $w_{i,I} \leq w_{i,J}$ for $J \subseteq I$ (consonance).
    \end{itemize}
  \end{minipage}
\end{tabular}
\end{frame}


\begin{frame}
  \frametitle{Example: Hierarchical LSD}
\begin{tabular}{ll}
  \begin{minipage}{.4\textwidth}
    \begin{tikzpicture}[scale=.5]
\node (Pl) at (25bp,-225bp)[draw,circle,fill=white] {$Pl$};
\node (Dl) at (175bp,-175bp)[draw,circle,fill=white] {$Dl$};
\node (Dh) at (225bp,-25bp)[draw,circle,fill=white] {$Dh$};
\draw [line width=1pt] (Pl.363) arc(273:290:209bp) node[fill=blue!20] {$H_{2}$} arc(290:302:209bp) to (Dl);
\draw [line width=1pt] (Pl.70) arc(160:145:282bp) node[fill=blue!20] {$H_{1}$} arc(145:112:282bp) to (Dh);
\draw [line width=1pt] (Dl.416) arc(326:340:212bp) node[fill=blue!20] {$H_{3}$} arc(340:355:212bp) to (Dh);
\end{tikzpicture}

  \end{minipage} &
  \begin{minipage}{.6\textwidth}
    \begin{itemize}
    \item 3 Hypotheses:
      \begin{enumerate}
      \item Placebo vs. High Dose
      \item Placebo vs. Low Dose
      \item High Dose vs. Low Dose
      \end{enumerate}
    \item Prefixed sequence: $H_1 \rightarrow H_2 \rightarrow H_3$
    \item Test $H_1$ at level $\alpha$
    \item If we reject $H_1$ test $H_2$ and $H_3$ at level $\alpha$
      simultaneously
    \item Can we generalize this?
    \end{itemize}
  \end{minipage}
\end{tabular}

\end{frame}

\begin{frame}
  \frametitle{Fixed sequence test of pairwise mean comparisons}
\begin{itemize}
\item {\bf Picture of hierarchical test hereabouts}
\end{itemize}

\begin{block}{Theorem}
  Assume that $M(\mathscr{H})$ neither contains loops nor
  parallels. Assume that hypotheses $H_1$ through $H_r$ have been
  rejected. Consider that hypotheses $H_{r+1}$ through $H_{r+t}$ are
  to be tested at level $\alpha_r$ according to a prefixed order.  If
  for some $1 \leq s \leq t$ holds
  \begin{equation}
    \label{eq:cond.hierarchical}
    \forall 1 \leq l < k \leq s: \; \exists i \leq r: i \sim \{(r+l),(r+k)\},
  \end{equation}
  hypotheses $H_{r+1}$ through $H_{r+s}$ can be tested simultaneously
  at full level $\alpha_ra$.
\end{block}

\end{frame}

\begin{frame}
  \frametitle{Example: Multiple doses}



\begin{tabular}{ll}
  \begin{minipage}{.4\textwidth}
    \begin{tikzpicture}[overlay]
      \node[yshift=70bp,xshift=-40bp] at (current page.south west){\begin{tikzpicture}[transform canvas={scale=.8},scale=.6]

\node (H1) at (75bp,-225bp)[draw,circle,fill=white] {$H1$};
\node (D1) at (225bp,-375bp)[draw,circle,fill=white] {$D1$};
\node (D2) at (375bp,-325bp)[draw,circle,fill=white] {$D2$};
\node (D3) at (375bp,-125bp)[draw,circle,fill=white] {$D3$};
\node (D4) at (225bp,-75bp)[draw,circle,fill=white] {$D4$};
\draw [line width=1pt] (H1) to (112bp, -262bp) node[fill=blue!20] {$H_{5}$} to (D1);
\draw [line width=1pt] (H1) to (150bp, -250bp) node[fill=blue!20] {$H_{3}$} to (D2);
\draw [line width=1pt] (H1) to (150bp, -200bp) node[fill=blue!20] {$H_{2}$} to (D3);
\draw [line width=1pt] (H1) to (112bp, -188bp) node[fill=blue!20] {$H_{1}$} to (D4);
\draw [line width=1pt] (D2) to (338bp, -337bp) node[fill=blue!20] {$H_{9}$} to (D1);
\draw [line width=1pt] (D3) to (338bp, -187bp) node[fill=blue!20] {$H_{7}$} to (D1);
\draw [line width=1pt] (D3) to (375bp, -175bp) node[fill=blue!20] {$H_{10}$} to (D2);
\draw [line width=1pt] (D4) to (225bp, -150bp) node[fill=blue!20] {$H_{6}$} to (D1);
\draw [line width=1pt] (D4) to (262bp, -137bp) node[fill=blue!20] {$H_{8}$} to (D2);
\draw [line width=1pt] (D4) to (262bp, -87bp) node[fill=blue!20] {$H_{4}$} to (D3);
\end{tikzpicture}
};
    \end{tikzpicture}
  \end{minipage} &
  \begin{minipage}{.6\textwidth}
    \begin{tikzpicture}[remember picture,overlay]
      \node[yshift=100bp,xshift=160bp] at (current page.north west){\begin{tikzpicture}[transform canvas={scale=.5}]

\node (H1) at (50bp,-350bp)[draw,circle split,fill=green!80] {$H1$ \nodepart{lower} $0$};
\node (H2) at (141bp,-350bp)[draw,circle split,fill=green!80] {$H2$ \nodepart{lower} $0$};
\node (H3) at (234bp,-350bp)[draw,circle split,fill=green!80] {$H3$ \nodepart{lower} $0$};
\node (H4) at (325bp,-350bp)[draw,circle split,fill=green!80] {$H4$ \nodepart{lower} $0$};
\node (H5) at (125bp,-475bp)[draw,circle split,fill=green!80] {$H5$ \nodepart{lower} $0$};
\node (H6) at (175bp,-475bp)[draw,circle split,fill=green!80] {$H6$ \nodepart{lower} $0$};
\node (H7) at (225bp,-475bp)[draw,circle split,fill=green!80] {$H7$ \nodepart{lower} $0$};
\node (H8) at (175bp,-625bp)[draw,circle split,fill=green!80] {$H8$ \nodepart{lower} $0$};
\node (H9) at (125bp,-625bp)[draw,circle split,fill=green!80] {$H9$ \nodepart{lower} $0$};
\node (H10) at (225bp,-625bp)[draw,circle split,fill=green!80] {$H10$ \nodepart{lower} $0$};

\draw [line width=1pt] (0bp,-315bp) -- (370bp,-315bp) -- (370bp,-385bp) -- (0bp,-385bp) -- (0bp,-315bp);
\node[anchor=north east] (text) at (370bp,-385bp) {reject all!};
\draw [line width=1pt] (80bp,-440bp) -- (270bp,-440bp) -- (270bp,-510bp) -- (80bp,-510bp) -- (80bp,-440bp);
\node[anchor=north east] (text) at (270bp,-510bp) {reject all!};
\draw [->,line width=1pt] (H1) to (96bp, -350bp) node[fill=blue!20] {$1$} to (H2);
\draw [->,line width=1pt] (H2) to (187bp, -350bp) node[fill=blue!20] {$1$} to (H3);
\draw [->,line width=1pt] (H3) to (278bp, -350bp) node[fill=blue!20] {$1$} to (H4);
\draw [->,line width=1pt] (175bp,-385bp) to (157bp, -420bp) node[fill=blue!20] {$1$} to (H5);
\draw [->,line width=1pt] (175bp,-385bp) to (175bp, -420bp) node[fill=blue!20] {$1$} to (H6);
\draw [->,line width=1pt] (175bp,-385bp) to (193bp, -420bp) node[fill=blue!20] {$1$} to (H7);
\draw [->,line width=1pt] (175bp,-510bp) to (175bp, -550bp) node[fill=blue!20] {$1$} to (H8);
\draw [->,line width=1pt] (175bp,-510bp) to (157bp, -550bp) node[fill=blue!20] {$1$} to (H9);
\draw [->,line width=1pt] (175bp,-510bp) to (193bp, -550bp) node[fill=blue!20] {$1$} to (H10);
\end{tikzpicture}
};
    \end{tikzpicture}
  \end{minipage}
\end{tabular}
\end{frame}

\begin{frame}
  \frametitle{General weighting strategies}
  \framesubtitle{QVA Example}


\begin{tabular}{ll}
  \begin{minipage}{.4\textwidth}
    \begin{tikzpicture}[remember picture,overlay]
      \node[yshift=-2.5cm,xshift=-1.4cm] at (current page.north west){\input{qvatest}};
    \end{tikzpicture}

  \end{minipage} &
  \begin{minipage}{.6\textwidth}
    \begin{tikzpicture}[remember picture,overlay]
      \node[yshift=-50bp,xshift=-180bp] at (current page.north east){\begin{tikzpicture}[transform canvas={scale=0.6},scale=.7]

\node (QVA) at (75bp,-275bp)[draw,circle,fill=white] {$QVA$};
\node (QAB) at (225bp,-225bp)[draw,circle,fill=white] {$QAB$};
\node (NVA) at (375bp,-275bp)[draw,circle,fill=white] {$NVA$};
\node (Pl) at (225bp,-75bp)[draw,circle,fill=white] {$Pl$};
\node (Tio) at (225bp,-375bp)[draw,circle,fill=white] {$Tio$};
\node (SE) at (75bp,-425bp)[draw,circle,fill=white] {$SE$};
\draw [line width=1pt] (QVA) to (112bp, -263bp) node[fill=blue!20] {$H_{4}$} to (QAB);
\draw [line width=1pt] (QVA) to (150bp, -275bp) node[fill=blue!20] {$H_{5}$} to (NVA);
\draw [line width=1pt] (QVA) to (112bp, -300bp) node[fill=blue!20] {$H_{6}$} to (Tio);
\draw [line width=1pt] (QVA) to (75bp, -350bp) node[fill=blue!20] {$H_{7}$} to (SE);
\draw [line width=1pt] (Pl) to (188bp, -125bp) node[fill=blue!20] {$H_{1}$} to (QVA);
\draw [line width=1pt] (Pl) to (225bp, -150bp) node[fill=blue!20] {$H_{2}$} to (QAB);
\draw [line width=1pt] (Pl) to (262bp, -125bp) node[fill=blue!20] {$H_{3}$} to (NVA);
\end{tikzpicture}
};
  \end{tikzpicture}
  \end{minipage}
\end{tabular}
\end{frame}

\begin{frame}
\frametitle{General weighting strategies}

\begin{enumerate}
\item Pretest using weighted Bonferroni or conventional graphical
  approach $R = \{i \in S: H_i \textrm{ rejected by pretest}\}$
\item Compute all matroid hyperplanes $\mathscr{H}$ for the matroid on
  the remaining hypotheses $I$
\item $R \gets R \cup \{i: p_i \leq \min_J w_{i,J} \alpha, J \in \mathscr{H}\}$
\item $A \gets \{\bigcup J: \min_j (p_j / w_{j,J}) > \alpha\}$ 
\item $I \gets I \setminus R$
\item Refine: $\mathscr{H} \gets \{J\in \mathscr{L}: \rho J = \rho
  \mathscr{H} - 1\}$ 
\item Prune: $\mathscr{H} \gets \{J \in \mathscr{H}: J \cap R = \emptyset\}$
\item Repeat until $\rho \mathscr {H} = 0$
\item Reject $R$ and $I \setminus A$
\end{enumerate}
\end{frame}
\section{Matriods from general families of hypotheses}



\begin{frame}
\frametitle{General Results}
\begin{itemize}
\item We give sufficient (and probably necessary) conditions for a
  general family of hypotheses $\mathscr{H}$ to define a matroid
\item We give sufficient (and probably necessary) conditions on
  $\mathscr{H}$ for $\mathscr{L}$ to be the set of strictly exhaustive
  sets
\item Systems with one sided hypotheses, and parallel hypotheses (\eg
  combined non-inferiority/superiority) do not give matroids with
  $\mathscr{L}$ all strictly exhaustive 
\end{itemize}


\end{frame}

\begin{frame}
  \frametitle{One sided hypotheses}
  \begin{block}{One-sided linear contrast hypotheses}
    A family of {\em one sided linear contrast hypotheses} is defined
    by a contrast matrix $\bs{A} \in \mathbb{R}^{m\times n}$ and
    vector $\bs{b} \in \mathbb{R}^n$ that define a  family of null
    hypotheses of the form $\mathscr{H} = \{H_i: \bs{A_i}^t\bs{\mu}
    \leq b_i,i \in \{1,...,n\}\}$, where $\bs{A}_i$ denotes the $i$-th
    column of $\bs{A}$ and $b_i$ the $i$-th componenth of $\bs{b}$.
  \end{block}
\end{frame}

\begin{frame}
  \frametitle{Hyperplane arrangement}
  \begin{itemize}
  \item Each hypothesis $H_i$ corresponds to a hyperplane $h_i =
    \{\bs{x} \in \mathbb{R}^m: \bs{A}_i^t\bs{x} = b_i\}$
  \item Each hyperplane $h_i$ divides $\mathbb{R}^m$ into two regions
    \begin{itemize}
    \item $h_i^+ = \{\bs{x} \in \mathbb{R}^m: \bs{A}_i^t\bs{x} <
      b_i\}$ 
    \item $h_i^- = \{\bs{x} \in \mathbb{R}^m: \bs{A}_i^t\bs{x}
      > b_i\}$
    \end{itemize}
  \item Defines a sign pattern 
    \begin{equation}
      \label{eq:sign}
      \delta_i(\bs{x}) = \left\{
        \begin{array}{ll}
          + & \text{if } \bs{x} \in h_i^+\\
          0 & \text{if } \bs{x} \in h_i \\
          - & \text{if } \bs{x} \in h_i^-\\
        \end{array}\right.
    \end{equation}
  \item $\delta_i$ is $-$ if $H_i$ false at $x$, or $0$ or $+$
    otherwise. $(\delta_1,...,\delta_n)$ tells us to which combination
    of true and false null hypotheses some $\bs{x}$ belongs. 

  \item Set of all sign patterns defined by $\mathscr{H}$:
    \begin{equation}
      \label{eq:faces}
      \mathcal{F}(\mathscr{H}) = \{\bs{\delta}(\bs{x}): \bs{x} \in \R^m\}
    \end{equation}
  \end{itemize}
\end{frame}

\begin{frame}
  \frametitle{Example}

\begin{tabular}{ll}
  \begin{minipage}{.4\textwidth}
  \includegraphics[height=1\textwidth]{spheresfukuda.eps}
  \end{minipage} &
  \begin{minipage}{.6\textwidth}
    \begin{itemize}
    \item 4 contrast hypotheses about 3 means
    \item Cut the hyperplanes with the sphere in $\R^3$ (makes
      visualisation easier)
    \item Cells of the sphere arrangment correspond to sign patterns
      of $\mathscr{H}$
    \item The abstraction of such an arrangement is an \alert{Oriented Matroid}
    \end{itemize}
  \end{minipage}
\end{tabular}

\end{frame}

\begin{frame}
  \frametitle{Stepwise multiple testing procedure for one-sided linear
    contrast hypotheses}

  Consider a family of general contrast hypotheses $\mathscr{H} =
  \{H_i: \bs{A_i}^t\bs{\mu} \leq b_i,i \in \{1,...,n\}\}$ and let
  $\mathscr{F}(\mathscr{H}) = \{ \bs{\delta}(\bs{x}): \bs{x} \in
  \R^m\}$ denote the corresponding set of sign patterns. Consider a
  set of decision rules $\varphi_F: \R^n \rightarrow \{0,1\}$ for
  which hold $P(\varphi_F = 1 | H_i: F_i \geq 0) \leq \alpha$. Then a
  corresponding decision rule for each elementary hypotheses $H_i$ is
  given by 
  \begin{equation}
    \label{eq:signtest}
    \psi_i = \min_{F: F_i \geq 0} \varphi_F,
  \end{equation}
  which controls the FWER at level $\alpha$ in the strong sense. 


\end{frame}


\begin{frame}
  \frametitle{Computation of $\mathscr{F}$}
  The question remains how to compute $\mathcal{F}$. For this observe
that the $h_i$ define a hyperplane configuration and $\mathcal{F}$ is
just the face lattice of the corresponding oriented
matroid. Consequently, $\mathcal{F}$ can be found using algorithms
for arrangement construction from polyhedral programming
\cite{Avis-KF-92,Avis-KF-96,Ferrez-KF-Liebling-01} which are
polynomial in time and input size and compact in space!

\end{frame}

\begin{frame}
  \frametitle{Example}
  3 ordered means 
\end{frame}

\section{Summary \& Outlook}



\begin{frame}
\frametitle{Summary}
\begin{itemize}
\item Matroids provide usefull abstraction for systems of algebraically dependent hypotheses
\end{itemize}
\end{frame}



\begin{frame}
\frametitle{Outlook}
\begin{itemize}
\item MCPmod on \sout{steroids} matroids
\end{itemize}
\end{frame}

\bibliographystyle{alpha}
\bibliography{\parentd paper/cleanrefs}
\end{document}