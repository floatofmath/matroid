% Created 2013-07-25 Thu 18:08
\documentclass[bigger]{beamer}
\usepackage[utf8]{inputenc}
\usepackage{tikz}
\usetikzlibrary{decorations,arrows,shapes}
\usepackage[T1]{fontenc}
\usepackage{fixltx2e}
\usepackage{graphicx}
\usepackage{longtable}
\usepackage{float}
\usepackage{wrapfig}
\usepackage{soul}
\usepackage{textcomp}
\usepackage{marvosym}
\usepackage{wasysym}
\usepackage{latexsym}
\usepackage{amssymb}
\usepackage{hyperref}
\usepackage{bm}
\tolerance=1000
\usepackage{amsmath} \usepackage{algorithm} \usepackage{graphicx}
\usepackage{amssymb} \usepackage{mathrsfs} 
\newcommand{\bs}[1]{\bm{#1}}
\newcommand{\ie}{{\em i.e.},~}
\newcommand{\eg}{{\em e.g.},~}
\providecommand{\alert}[1]{\textbf{#1}}

\title{Matroids from families of null hypotheses}
\author{Willi Maurer, Florian Klinglmueller}
\date{\today}
\hypersetup{
  pdfkeywords={},
  pdfsubject={},
  pdfcreator={Emacs Org-mode version 7.8.11}}

\begin{document}

\maketitle



\begin{frame}
\frametitle{Outline}
\setcounter{tocdepth}{3}
\tableofcontents
\end{frame}




\section{Logically dependent hypotheses}


\begin{frame}
\frametitle{Pairwise mean comparisons (two-sided)}
\begin{itemize}

\item Consider $\mu_i, i \in \{1,...,m\}$

\item $H_k: \mu_i = \mu_j, k \in S = \{1,...,n\}$, where $n \leq m*(m-1)/2$

\item Can be represented by an undirected graph $\mathscr{G} = (V,E)$:
\begin{itemize}
\item vertices correspond to means, \emph{i.e.}, $V = \{\mu_1,...,\mu_m\}$,
\item edges correspond to hypotheses, \emph{i.e.}, $E =
      \left\{e_1,...,e_n\} \right\}$, $e_l = \{\mu_i,\mu_j\}$, $i \neq
      j$, and $e_l \neq e_k$ for $l \neq k$
\end{itemize}
\item Let $H_I = \bigcap_{k \in I} H_k$, $I \subseteq S$
\item $I$ induces a subgraph $\mathscr{G}_I$ on the edge set $E_I=\{e_i: i
  \in I\}$
\end{itemize} % ends low level






\begin{frame}
\frametitle{Example 1: 4 Means}
\begin{tikzpicture}[scale=1]
\node (H1) at (225bp,-125bp)[draw,circle,fill=white] {$\mu_1$};
\node (H2) at (425bp,-125bp)[draw,circle,fill=white] {$\mu_2$};
\node (H3) at (425bp,-325bp)[draw,circle,fill=white] {$\mu_3$};
\node (H4) at (225bp,-325bp)[draw,circle,fill=white] {$\mu_4$};
\draw [line width=1pt] (H1) to (325bp, -124bp) node[fill=blue!20] {$H_{1}$} to (H2);
\draw [line width=1pt] (H1) to (275bp, -175bp) node[fill=blue!20] {$H_{2}$} to (H3);
\draw [line width=1pt] (H2) to (425bp, -224bp) node[fill=blue!20] {$H_{4}$} to (H3);
\draw [line width=1pt] (H3) to (326bp, -324bp) node[fill=blue!20] {$H_{6}$} to (H4);
\draw [line width=1pt] (H4) to (225bp, -227bp) node[fill=blue!20] {$H_{3}$} to (H1);
\draw [line width=1pt] (H4) to (374bp, -176bp) node[fill=blue!20] {$H_{5}$} to (H2);
\end{tikzpicture}

\end{frame}



\begin{frame}
\frametitle{General contrast tests (two-sided)}
\begin{itemize}

\item Consider $m$-variate random variable $\bs{X} \sim F(\bs{\mu})$, $\bs{\mu} = (\mu_1,...,\mu_m)^t$

\item Contrasts $\bs{c}_i = (c_{i1},...,c_{im})^t$, $\sum_{j = 1}^m
  c_{ij} = 0$
\item Contrast matrix  $\bs{C} = \left( \bs{c}_1,...,\bs{c}_n\right)$,
  with sub-matrix $\bs{C}_I = \left( \bs{c}_i \right)_{i \in
    I}$
\item Elementary hypotheses $H_i: \bs{c}_i^t\bs{\mu} = 0$,
\item Intersection hypotheses $H_I: \bs{C}_I^t\bs{\mu} = 0$

\item Elementary hypotheses are represented by column vectors,
  intersection hypotheses by sub-matrices of the contrast matrix

\end{itemize} % ends low level
\end{frame}



\begin{frame}
\frametitle{Example 2: Contrast test}
\begin{itemize}

\item \textbf{/Some illustration hereabouts/}

\end{itemize} % ends low level
\end{frame}



\begin{frame}
\frametitle{Dependent null hypotheses}

\begin{itemize}
\item $H_i$ depends on $H_I$ iff $H_{I \cup i} = H_{I}$, induces a
  dependence relation $\sim_H$ on $\mathcal{P}(S)$
\item $H_I$ independent iff for all $i \in I$, $i \nsim_H I$,
\item $I$ independent $\Leftrightarrow$ $H_{I} \subsetneq  H_{I\setminus i}$
\item For pairwise comparisons $I$ independent iff there are no cycles in $\mathscr{G}_I$
\item For general contrast tests $I$ independent iff $\bs{C_I}$ has full
  column rank
\end{itemize} % ends low level
\end{frame}



\section{Matroids}



\begin{frame}
\frametitle{Matroids}
\begin{itemize}


\item A matroid on $S$ endows $\mathcal{P}(S)$ with an algebraic
  structure 
\item Matroids generalize the structures of graphs and vector spaces,
  \eg circuits, bases, independent sets
\item Many ways to define a matroid
  \begin{itemize}
  \item From a graph perspective: Circuit axioms
  \item From a vector space perspective: Independence axioms
  \end{itemize}
\item There is a dual matroid $M^*$ which can be helpful for proofs and algorithms 
\item \textbf{/Some illustration hereabouts/}
\end{itemize} % ends low level
\end{frame}




% \begin{frame}
% \frametitle{Circuit axioms}

% Consider $S$ and a collection of subsets $\mathscr{C}$ for which holds
% \begin{enumerate}
% \item $C_1 \neq C_2 \in \mathscr{C}$, then $C_1 \nsubseteq C_2$,
% \item $C_1 \neq C_2 \in \mathscr{C}$, and $i \in C_1 \cap C_2$ then
%    $\exists C_3 \subseteq (C_1 \cup C_2)$,
% \end{enumerate}
% then $\mathscr{C}$ is the set of circuits of a matroid $M$ on $S$

% \begin{itemize}

% \item \textbf{/Some illustration hereabouts/}

% \end{itemize} % ends low level
% \end{frame}


\begin{frame}
\frametitle{Independence axioms}
\begin{block}{Independence}
  Consider $S$ and a collection of subsets $\mathscr{I}$ for which
  holds
  \begin{enumerate}
  \item $\emptyset \in \mathscr{I}$,
  \item $I \in \mathscr{I}$ and $J \subseteq I$ then $J \in
    \mathscr{I}$,
  \item $I,J \in \mathscr{I}$, $|I| = |J + 1|$ then $\exists j \in I
    \setminus J$ such that $J \cup i \in \mathscr{I}$,
  \end{enumerate}
  then $\mathscr{I}$ is the set of independent sets of a matroid $M$
  on $S$. This defines an independence relation $i \nsim_M I$ iff
  $I\cap i \in \mathscr{I}$. 
\end{block}

\begin{block}{Circuit}
  $C \subset S$ is a circuit iff $C \neq \mathscr{I}$ and for all $I
  \subset C$, $I \in \mathscr{I}$ (\ie minimally dependent set)
\end{block}
\end{frame}


\begin{frame}
\frametitle{Dependence axioms}

\begin{block}{Dependence}
  Consider $S$ and a relation $\sim_M$ on elements and subsets of $S$,
  then if for any $x,y_1,...,y_n,z_1,...,z_m$ holds
  \begin{enumerate}
  \item $y_i \sim \{y_1,...,y_n\}$ for all $i \leq n$
  \item if $n\geq 1$, $x \sim_M \{y_1,...,y_n\}$ and $x \nsim_M
    \{y_1,...,y_{n-1}$ then $y_n \sim \{y_1,...,y_{n-1},x\}$
  \item if $x \sim_M \{y_1,...,y_n\}$ and for all $i \leq n$ $y_i
    \sim_M \{z_1,...,z_m\}$ then $x \sim_M \{z_1,...,z_m\}$
  \end{enumerate}
  then $\sim_M$ defines a dependence relation of a matroid on $S$.
\end{block}

\end{frame}

\begin{frame}{Maximally independent sets}
  \begin{itemize}
  \item Bases are maximally independent sets of $S$
  \item Rank $\rho I$ is the cardinality of a maximally indepedendent set in $I$
  \item Circuits are minimally dependent sets, \ie each subset of $C$
    with rank $\rho C - 1$ is a maximally independent set of $C$
  \end{itemize}
\end{frame}

\begin{frame}
\frametitle{Maximally dependent sets}
\begin{itemize}
\item The closure of a set $\sigma I = \{i \in S: i \sim I\}$
\item Closed sets {\em flats}: $\mathscr{L} = \{I \subseteq S: \sigma I = I\}$
\item Hyperplanes are flats of rank $\rho S - 1$
\item All flats are given by closing hyperplanes under intersection
\item Circuits in $M^*$ are the complements of hyperplanes in $M$
\end{itemize} % ends low level
\end{frame}



\begin{frame}{Examples}
\framesubtitle<1>{Independent set}  
\framesubtitle<2>{Base}  
\framesubtitle<4>{Dependent set}  
\framesubtitle<3>{Cycle}  
\framesubtitle<5>{Closure}  
\only<1>{\begin{tikzpicture}[scale=.5]
\node (H1) at (225bp,-125bp)[draw,circle,fill=white] {$\mu_1$};
\node (H2) at (425bp,-125bp)[draw,circle,fill=white] {$\mu_2$};
\node (H3) at (425bp,-325bp)[draw,circle,fill=white] {$\mu_3$};
\node (H4) at (225bp,-325bp)[draw,circle,fill=white] {$\mu_4$};
\draw [line width=1pt] (H1) to (325bp, -124bp) node[fill=blue!20] {$H_{1}$} to (H2);
\draw [line width=1pt] (H2) to (425bp, -224bp) node[fill=blue!20] {$H_{4}$} to (H3);
\end{tikzpicture}
}
\only<2>{\begin{tikzpicture}[scale=.5]
\node (H1) at (225bp,-125bp)[draw,circle,fill=white] {$\mu_1$};
\node (H2) at (425bp,-125bp)[draw,circle,fill=white] {$\mu_2$};
\node (H3) at (425bp,-325bp)[draw,circle,fill=white] {$\mu_3$};
\node (H4) at (225bp,-325bp)[draw,circle,fill=white] {$\mu_4$};
\draw [line width=1pt] (H1) to (325bp, -124bp) node[fill=blue!20] {$H_{1}$} to (H2);
\draw [line width=1pt] (H2) to (425bp, -224bp) node[fill=blue!20] {$H_{4}$} to (H3);
\draw [line width=1pt] (H4) to (225bp, -227bp) node[fill=blue!20] {$H_{3}$} to (H1);
\end{tikzpicture}
}
\only<4>{\begin{tikzpicture}[scale=.5]
\node (H1) at (225bp,-125bp)[draw,circle,fill=white] {$\mu_1$};
\node (H2) at (425bp,-125bp)[draw,circle,fill=white] {$\mu_2$};
\node (H3) at (425bp,-325bp)[draw,circle,fill=white] {$\mu_3$};
\node (H4) at (225bp,-325bp)[draw,circle,fill=white] {$\mu_4$};
\draw [line width=1pt] (H1) to (325bp, -124bp) node[fill=blue!20] {$H_{1}$} to (H2);
\draw [line width=1pt] (H1) to (275bp, -175bp) node[fill=blue!20] {$H_{2}$} to (H3);
\draw [line width=1pt] (H2) to (425bp, -224bp) node[fill=blue!20] {$H_{4}$} to (H3);
\draw [line width=1pt] (H4) to (225bp, -227bp) node[fill=blue!20] {$H_{3}$} to (H1);
\end{tikzpicture}
}
\only<3>{\begin{tikzpicture}[scale=.5]
\node (H1) at (225bp,-125bp)[draw,circle,fill=white] {$\mu_1$};
\node (H2) at (425bp,-125bp)[draw,circle,fill=white] {$\mu_2$};
\node (H3) at (425bp,-325bp)[draw,circle,fill=white] {$\mu_3$};
\node (H4) at (225bp,-325bp)[draw,circle,fill=white] {$\mu_4$};
\draw [line width=1pt] (H1) to (325bp, -124bp) node[fill=blue!20] {$H_{1}$} to (H2);
\draw [line width=1pt] (H1) to (275bp, -175bp) node[fill=blue!20] {$H_{2}$} to (H3);
\draw [line width=1pt] (H2) to (425bp, -224bp) node[fill=blue!20] {$H_{4}$} to (H3);
\end{tikzpicture}
}
\only<5>{\begin{tikzpicture}[scale=.5]
\node (H1) at (225bp,-125bp)[draw,circle,fill=white] {$\mu_1$};
\node (H2) at (425bp,-125bp)[draw,circle,fill=white] {$\mu_2$};
\node (H3) at (425bp,-325bp)[draw,circle,fill=white] {$\mu_3$};
\node (H4) at (225bp,-325bp)[draw,circle,fill=white] {$\mu_4$};
\draw [line width=1pt] (H1) to (325bp, -124bp) node[fill=blue!20] {$H_{1}$} to (H2);
\draw [line width=1pt] (H1) to (275bp, -175bp) node[fill=blue!20] {$H_{2}$} to (H3);
\draw [line width=1pt] (H2) to (425bp, -224bp) node[fill=blue!20] {$H_{4}$} to (H3);
\draw [line width=1pt] (H3) to (326bp, -324bp) node[fill=blue!20] {$H_{6}$} to (H4);
\draw [line width=1pt] (H4) to (225bp, -227bp) node[fill=blue!20] {$H_{3}$} to (H1);
\draw [line width=1pt] (H4) to (374bp, -176bp) node[fill=blue!20] {$H_{5}$} to (H2);
\end{tikzpicture}
}
\end{frame}

\section{Matroids from logically dependent hypotheses}



\begin{frame}
\frametitle{Matroids from logically dependent hypotheses}

  \begin{block}{Theorem}
    For a family of two-sided pairwise mean comparisons $H_i$, $i \in
    S$, an intersection hypotheses $H_I$ is independent iff the
    corresponding edge set $\{e_i:i \in I\}$ is independent in the
    cycle matroid on $\mathscr{G}_I$.  

For a family of two-sided
    general contrast tests $H_i$, $i \in S$, an intersection
    hypotheses $H_I$ is independent iff the corresponding contrast
    matrix $\bs{C}_I$ is independent in the vectorial matroid on
    $\mathbb{R}^m$.

    \begin{displaymath}
      i \sim_H I \Leftrightarrow i \sim_M I
    \end{displaymath}
  \end{block}


\end{frame}



\begin{frame}
\frametitle{Closed test procedure}
\begin{itemize}

\item The flats of a matroid are closed under intersection

\item Every elementary hypothesis is a flat

\item Using local level-$\alpha$ tests for each flat we have a closed test procedure that controls the FWER in the strong sense (Closed test principle)

\end{itemize} % ends low level
\end{frame}



\begin{frame}
\frametitle{Examples}
\begin{itemize}

\item Pairwise comparisons

\item The flats of the complete graph correspond to the partitions of $\{1,...,m\}$

\item The flats of a matroid generate a lattice, procedures testing not all possible comparisons correspond to intervals on that lattice

\end{itemize} % ends low level
\end{frame}



\begin{frame}
\frametitle{Connections to existing procedures}

\textbf{Old wine in new bottles ((c) L. Hothorn)}
\begin{itemize}

\item Flats <-> Weakly Exhaustive set (Hommel, Bergmann)

\item Lemma 1.5 in (Bernhard) about weakly exhaustive sets are ``closure axioms'' of a matroid

\item Graphical procedures for pairwise comparisons (Weichert) identify dependence using cycle in the graph


\item \emph{However:}
\begin{itemize}
\item proofs are shorter (more elegant),
\item efficient algorithms exist to find the flats of matroids,
\item we may define procedures using independent sets,
\item \ldots{}
\end{itemize}

\end{itemize} % ends low level
\end{frame}


\begin{frame}{Matroid procedures for pairwise mean comparisons}

  \begin{itemize}
  \item Two-sided comparisons between $m$ means
  \item The cycle matroid of the corresponding graph representation
    provides an equivalent dependence structure
  \item The flats of a (simple) complete graph are
    isomporphic to the partitions of a set of size $m$
  \item Provides an efficient way to generate all exhaustive sets by
    enumeration of all partitions [ER '88] or hyperplanes 
  \item The number of exhaustive sets are given by the Bell numbers
    
  \end{itemize}
  
\end{frame}

\section{Graphical approaches for logically dependent hypotheses}



\begin{frame}
\frametitle{Reminder: Graphical approaches}
\end{frame}



\begin{frame}
\frametitle{Hierarchical tests}
\end{frame}



\begin{frame}
\frametitle{General weighting strategies}

\begin{enumerate}
\item Pretest using weighted Bonferroni or conventional graphical
  approach $R = \{i \in S: H_i \textrm{ rejected by pretest}\}$
\item Compute all matroid hyperplanes $\mathscr{H}$ for the matroid on
  the remaining hypotheses $I$
\item $R \gets R \cup \{i: p_i \leq \min_J w_{i,J} \alpha, J \in \mathscr{H}\}$
\item $A \gets \{\bigcup J: \min_j (p_j / w_{j,J}) > \alpha\}$ 
\item $I \gets I \setminus R$
\item Refine: $\mathscr{H} \gets \{J\in \mathscr{L}: \rho J = \rho
  \mathscr{H} - 1\}$ 
\item Prune: $\mathscr{H} \gets \{J \in \mathscr{H}: J \cap R = \emptyset\}$
\item Repeat until $\rho \mathscr {H} = 0$
\item Reject $R$ and $I \setminus A$
\end{enumerate}
\end{frame}
\section{Matriods from general families of hypotheses}



\begin{frame}
\frametitle{Matroid conditions}
\end{frame}



\begin{frame}
\frametitle{Flats <-> Weakliy exhaustive sets}
\end{frame}



\begin{frame}
\frametitle{Flats <-> Strictly exhaustive sets}
\end{frame}
\section{Summary \& Outlook}



\begin{frame}
\frametitle{Summary}
\end{frame}



\begin{frame}
\frametitle{Outlook}

\end{frame}

\end{document}