\documentclass[a4paper,12pt]{article}
\usepackage{algorithmic}
\usepackage{algorithm}
\usepackage{graphicx}
%\usepackage{titlesec}
\usepackage{multirow}
\setlength\parindent{0in}
\usepackage[hdivide={2cm,*,2cm},vdivide={1in,*,1in}]{geometry}
\usepackage{setspace}
\doublespacing
\usepackage{amsmath}
\usepackage{amssymb}
\usepackage{mathrsfs}
\newcommand{\ud}{\, {\rm d} \kern-.015em }
\newtheorem{theorem}{Theorem}
\newtheorem{corollary}{Corollary}
\newtheorem{comment}{Comment}
\newtheorem{lemma}{Lemma}
\newtheorem{definition}{Definition}
\newtheorem{proposition}{Proposition}
\newtheorem{bsp}{Example}
\usepackage{natbib}
\usepackage{color}
\newenvironment{proof}[1][Proof]{\noindent\textbf{#1:} }{}
\newcommand{\bs}[1]{\boldsymbol{#1}}
\usepackage{authblk}
\usepackage{lineno}
%\titlelabel{\thetitle.\quad}
\newcommand{\gmcp}{\mathcal{G}}

\renewcommand{\thesection}{\arabic{section}.}
\renewcommand{\thesubsection}{\arabic{section}.\arabic{subsection}}
%% Comments
%\newcommand{\fcomment}[1]{ } % Turn off
\newcommand{\fcomment}[1]{\footnote{{\color{red} #1}}} % Turn on
\newcommand{\ie}{{\em i.e.,}~}
\newcommand{\indf}[1]{\mathbf{1}_{\left\{#1\right\}}}
%\newcommand{\text}[1]{\textrm{#1}}
\newcommand{\R}{\mathbb{R}}

\title{Multiple testing with logical dependencies}

\author{Florian Klinglmueller, Willi Maurer}

%\affil[1]{Medical and Pharmaceutical Statistics Research Unit,
%Lancaster University, Lancaster, United Kingdom}
%\affil[2]{European Medicines Agency, London, United Kingdom}
%\affil[3]{Center for Medical Statistics, Informatics and Intelligent Systems, Medical University of Vienna, Vienna, Austria%}
%\affil[*]{{\small To whom correspondence should be
%addressed: E-mail: florian.klinglmueller@meduniwien.ac.at}}

\date{}

\begin{document}


\maketitle

\section{Introduction}

Closed testing procedures provide powerful multiple tests for a wide
array of general multiple comparison problems. Recently, methods based
on closed tests of (weighted) Bonferroni tests have gotten much
attention in the literature. They do not require any assumptions on or
knowledge about the correlation structure between test statistics. In
combination with weighting strategies such procedures may be tailored
to reflect contextual relations and differences between the elementary
hypotheses in practical applications. Graphical methods are available
that offer a visually accessible tool to design and communicate such
procedures.

\section{Examples}

\subsection{Pairwise comparisons of population means}
\begin{itemize}
\item All pairwise
\item Gold standard designs
\item Change point analysis
\item General treatment control - treatment treatment situations
\end{itemize}

\subsection{General linear contrast tests}
\begin{itemize}
\item Cost effectiveness (Georg)
\item SNP Studies (Hothorn-Hothorn)
\item Order restriction contrasts (Williams, ...)
\end{itemize}

\subsection{Further examples}

\begin{description}
\item[Test for the degree of a polynomial]
\item[Model selection]
\item[Compositional data]
\item[Monotonicity restrictions not arising from linear dependencies]
\end{description}

\section{Two sided hypotheses}

\subsection{Some observations on pairwise comparisons}

\begin{definition}
  Let $X$ be a random variable taking values in $\mathbb{R}^m$ with
  multivariate distribution $F(\bs{\mu})$ having mean vector $\bs{\mu}
  \in \mathbb{R}^m$. Let $R \subset \{1,...,m\}^2$, such that $i <
  j$ for all $(i,j) \in R$. For $(i,j) \in R$ $H_{ij} = \mu_i =
  \mu_j$. We call the family of elementary hypotheses $\dot{R}
  = \{H_{ij}: (i,j) \in R\}$ a {\bf two-sided pairwise comparison
    problem. (tPCP)} Let $\dot{\mathcal{T}}(m)$ denote the set of all
  possible pairwise comparison problems of $m$ means.
\end{definition}

\begin{bsp}
  Consider $X \sim N((\mu_1,\mu_2,\mu_3,\mu_4),\sigma \mathbb{I}_4)$
  let $R = \{(1,2),(1,3),(1,4)\}$ and $S =
  \{(1,4),(2,4),(3,4),(1,3),(1,2),(2,3)\}$. Then the
  corresponding tPCP's $\dot{R},\dot{S} \in \dot{\mathcal{T}}(4)$.
\end{bsp}

\begin{definition}
  A two-sided pairwise comparison problem $\dot{S}$ defines an
  undirected graph $G(\dot{S}) = \{M,T\}$ with Nodes $M = \{1,...,m\}$
  and Edges $T$. We call $G(\dot{S})$ the {\bf comparigram}. We denote
  by $V(G(\dot{S})$, $E(G(\dot{S})$ the vertex and edge sets of the
  corresponding comparigram. If clear from the context we may just use
  $E(\dot{S}),V(\dot{S})$. 
\end{definition}

\begin{bsp}
  $\dot{R}$ in above Example is represented by a claw with 3 edges,
  $\dot{S}$ is represented by the complete graph $K_4$ with 4 vertices.
\end{bsp}

\begin{definition}
  Let $h_{lk}:  M^2 \rightarrow \dot{\mathcal{T}}(m)$,
  $(l,k) \mapsto \{H_{ij}: (i,j) \in T\setminus \{(l,j)\}\}$ define the
  {\bf rejection operator}. We write $h_{lk} \circ \dot{S} =
  \dot{S}\setminus (l,k)$. Clearly   $h_{a} \circ h_{b} \circ \dot{S}
  = h_{b} \circ h_{a} \circ \dot{S}$.
\end{definition}

\begin{definition}
  Consider $\dot{R},\dot{S} \in \dot{\mathcal{T}}(m)$ with associated
  comparigrams $L = G(\dot{R}), K = G(\dot{R})$ respectively. We say
  that $\dot{R}$ {\bf isomorphic} to $\dot{S}$, $\dot{R} \cong
  \dot{S}$ if $L$ is isomorphic to $K$, that is there exists a
  bijection between nodes of the graphs $f:V(K) \rightarrow V(L)$, $a
  \mapsto u$ such that for $a,b \in V(K)$, $f(a)$ is adjacent to
  $f(b)$, if and only if $a$ and $b$ are adjacent to each other.
\end{definition}

\begin{definition}
  Consider $\dot{S} \in \dot{\mathcal{T}}(m)$ and $h_a,h_b$, $a,b \in
  E(S)$ we say that $h_a \cong h_b$ are {\bf equivalent rejections} if $h_a
  \circ \dot{S} \cong h_b \circ \dot{S}$.
\end{definition}

\begin{definition}
  Let $h = (h_1,...,h_n)$ whith $h_i \in T$ and $n \leq |T|$ define a
  {\bf fixed rejection sequence} of length $n$.
\end{definition}

\begin{definition}
  Consider PCP $\dot{S}$. Two fixed rejection sequences $h, e$ are
  called {\bf equivalent} if $h_1\circ ... \circ h_i \circ \dot{S} \cong e_1 \circ
  ... \circ e_i \circ \dot{S}$ for all   $i \in \{1,...,n\}$
\end{definition}
\begin{definition}
  Let $\dot{K}_n$ denote the $tPCP$ consisting of all pairwise
  comparisons of $n$ different means.
\end{definition}

\begin{theorem}
  Properties of $\circ$:
  \begin{enumerate}
    \item $\dot{S} \circ e_1 \circ e_2 \dot{S} \circ e_2 \circ e_1$
    \item For $e,f,u,v \in E(\dot{S})$, such that $\dot{S} \circ e
      \cong \dot{S} \circ f$, and $\dot{S} \circ e \circ u \cong
      \dot{S} \circ u \circ v$ then $\exists z \in E(\dot{S} \circ f)$
      such that $\dot{S} \circ f \circ z \cong \dot{S} \circ e \circ
      u$. 
  \end{enumerate}
\end{theorem}

\begin{theorem}
  The tPCP $\dot{K}_4$ has 6 distinct (up to isomorphism) fixed
  rejection sequences. Figure \ref{fig:k4} shows a graphical
  representation of the distinct rejection sequences for
  $K_4$. $\dot{K}_5$ has $657$ and $\dot{K}_6$ has $4836277$ it seems
  that this sequence grows in the order of $\mathcal{O}(e^{n^2})$.
\end{theorem}

\begin{proof}
  Proof by exhaustive computer search.
\end{proof}

\begin{figure}

  \caption{Rejection tree for $\dot{K}_4$}
  \label{fig:k4}
\end{figure}

\begin{theorem}
  $\dot{K}_4$ can be represented by the complete graph $K_4$ with 4
  vertices and 6 edges. The resulting null hypotheses are vertices of
  the line-graph $\mathcal{L}{K_4}$. $\mathcal{L}{K_4}$ is an
  octahedron. The exhaustive sets of $\dot{K_4}$ correspond to
  $\mathcal{L}(K_4)$, the 3 faces of the octahedron with differently
  labeled edges representing cliques of size 3 in $K_4$, the 3 edges
  in the complementary graph of $\mathcal{L}(K_4)$, and the $6$
  vertices themselves.
\end{theorem}


\subsection{Connections to matroid theory}

In this section we first show that any (finite (?)) system of null
hypotheses together with its exhaustive sets defines a matroid. The
closure operator of that matroid is the function that for a given
index set assigns the smallest containing exhaustive index set. We
then focus on the specific problem of embedding two-sided pairwise
comparisons in matroid theory. We show that the exhaustive sets of a
tPCP are flats in the cycle matroid of the comparigram, and vice
versa. 


Let $X$ denote a random variable describing the outcome of an
experiment, e.g., the results of a randomized clinical trial, with
distribution $F^\theta$ from parametric family $\mathcal{F} =
{F_\vartheta|\vartheta \in \Theta}$. Let $\mathcal{H} \ \{H_i: \theta
\in \Theta_i, \Theta_i \subset \Theta, i \in S\}$ denote a family of
null hypotheses about true parameter $\theta$ with associated index
set $S = \{1,...,n\}$. Let $H_I : \theta \in \bigcap_{i \in I}
\Theta_i$ denote the intersection hypothesis associated with $I$, \ie
that assumes that all $H_i$, $i \in I$ are true simultaneously. For
completeness define $H_\emptyset : \theta \in \Theta$. To shorten
notation we will write $H_I = \bigcap_I \Theta_i$. We define a subset
$I \in S$ to be independent if the corresponding intersection
hypothesis is altered if any index is removed from $I$, that is

\begin{definition}{\bf Independent index set}
\label{def:indep}
  Let $I \subseteq S$. We say that $I$ is independent if $\bigcap_{i
    \in I} \Theta_i \subsetneq \bigcap_{i \in I\setminus \{j\}}
  \Theta_i$ for all $j \in I$. 
\end{definition}

Unfortunately this rather simple definition of independence is not
enough such that the system of intersection hypotheses is a
matroid. An additional requirement is needed.


\begin{theorem}
  Consider a family of hypotheses $\mathcal{H} = \{H_i:\theta \in
  \Theta_i,i \in S\}$. Let $\mathscr{I} = \{I \subseteq S: I \text{
    independent}\}$ denote the set of independent sets of
  $\mathcal{H}$. If for all $I,J \in \mathscr{I}$ with $|I| = |J|+1$,
  holds 
  \begin{align}
    \label{cond:matroida}
    \bigcap_J \Theta_i &\nsubseteq \bigcap_{I\setminus J} \Theta_i,
    \text{ and} \\
    \label{cond:matroidb}
    \bigcap_{J}\Theta_i &\nsupseteq     \bigcup_{I \setminus J} \Theta_i ,
  \end{align}
  then $\mathscr{I}$ defines a matroid $M(\mathcal{H})$ over $S$.
\end{theorem}

\begin{proof}
  We need to show the independence axioms:
  \begin{enumerate}
  \item $\emptyset \in \mathscr{I}$
  \item $I  \in \mathscr{I}: J \subset I \Rightarrow J \in \mathscr{I}$
  \item $I,J \in \mathscr{I}$: $|I|=|J|+1 \Rightarrow \exists j \in
    I\setminus J: J \cup j \in \mathscr{I}$ 
  \end{enumerate}
  The first axiom follows directly from $\Theta_i \neq \Theta$ and
  $H_\emptyset = \Theta$. 

  To show the second axiom let, without loss of generality, $I =
  \{1,....,k\}$ $J = \{1,...,l\}$ $l \leq k$. We use induction. For $l
  = k$, $I = J$ and the result is obviously true. For $l = k-1$ assume
  there is some $j \in J$, such that $\bigcap_{J \setminus j} \Theta_i
  = \bigcap_{J} \Theta_i$. Since $I \in \mathscr{I}$, we have
  $\bigcap_{I} \Theta_i \subsetneq \bigcap_{J} \Theta_i$ and
  $\bigcap_I \Theta_i \subsetneq \bigcap_{I \setminus j} \Theta_i =
  \Theta_k \cap \bigcap_{J \setminus j} \Theta_i= \Theta_k \cap \bigcap_{J}
  \Theta_i= \bigcap_I \Theta_i$. Assume that it holds for $l = k - m$
  then using the same arguments it follows that it also holds for $l =
  k - (m + 1)$.

  To show the third axiom some additional assumptions on the sets
  $\Theta_i$ are required. We need to show that for $I,J \in
  \mathscr{I}$ there is a $j \in I \setminus J$ such that $\bigcap_{J
    \cup j}\Theta_i \subsetneq \bigcap_{(J\cup j)\setminus l}
  \Theta_i$ for all $l \in (J \cup j)$. One can write down necessary
  and sufficient conditions for the exchange axiom by enumeration of
  all possible cases of $I,J$. First observe that the axiom holds for
  $J \subset I$ with $l = I \setminus J$ otherwise consider $0 < l <
  k$ and $J = \{i_1,...,i_k\}$ and $I =
  \{i_{k+1-l},...,i_{2k-l+1}\}$. Then assuming that the exchange axiom
  does not hold for $k = 1$, $l = 1$, \ie $J = \{i_1\}$
  and $I =   \{i_2,i_3\}$, we have $\Theta_{i_1} \cap \Theta_{i_2} = \Theta_{i_1}$
  or $\Theta_{i_1} \cap \Theta_{i_2} = \Theta_{i_2}$, and $\Theta_{i_1} \cap
  \Theta_{i_3} = \Theta_{i_1}$ or $\Theta_{i_1} \cap \Theta_{i_3} = \Theta_{i_3}$. Now $\Theta_{i_1} \cap \Theta_{i_2} = \Theta_{i_1}$ that is
  $\Theta_{i_2} \subseteq \Theta_{i_1}$ and $\Theta_{i_1} \cap \Theta_{i_3} =
  \Theta_{i_3}$, \ie $\Theta_{i_1} \subseteq \Theta_{i_3}$, gives $\Theta_{i_2}
  \subseteq \Theta_{i_3}$ which contradicts the independece of $i_2$
  and $i_3$, and the same holds vice versa. Consequently either
  $\Theta_1 \subseteq \Theta_2$ and $\Theta_1 \subseteq \Theta_3$ or
  $\Theta_2 \subseteq \Theta_1$ and $\Theta_3 \subseteq \Theta_1$. Now
  a sufficient and necessary condition for the former is $\Theta_1 \in
  \Theta_2 \cap \Theta_3$ and for the latter $\Theta_2 \cup \Theta_3
  \subset \Theta_1$. As these conditions are necessary and sufficient
  negating them provides necessary and sufficient conditions for the
  exchange axiom that is:
  \begin{description}
  \item[$k=1,l=1$:] $\Theta_{i_1} \nsubseteq \Theta_{i_2} \cap
    \Theta_{i_3}$ and $\Theta_{i_2} \cup \Theta_{i_3} \nsubseteq \Theta_{i_1}$
  \item[$k=2,l=1$:] $\Theta_{i_1} \cap \Theta_{i_2} \nsubseteq
    \Theta_{i_3} \cap \Theta_{i_4}$ and $\Theta_{i_3} \cup \Theta_{i_4} \nsubseteq \Theta_{i_1} \cap \Theta_{i_2}$
  \item[$k=2,l=0$:] $\Theta_{i_1} \cap \Theta_{i_2} \nsubseteq
    \Theta_{i_3} \cap \Theta_{i_4} \cap \Theta_{i_5}$ and
    $\Theta_{i_3} \cup \Theta_{i_4} \cup \Theta_{i_5} \nsubseteq \Theta_{i_1} \cap \Theta_{i_2}$
  \item[$k=3,l=2$:] $\Theta_{i_1} \cap \Theta_{i_2} \cap \Theta_{i_3}\nsubseteq
    \Theta_{i_4} \cap \Theta_{i_5}$ and $\Theta_{i_4} \cup
    \Theta_{i_5} \nsubseteq \Theta_{i_1} \cap \Theta_{i_2} \cap \Theta_{i_3}$
  \item[....]
  \item[General $k,l$:] $\bigcap_{r \leq k-l} \Theta_{i_r} \nsubseteq
    \bigcap_{k+1 \leq s \leq 2k-l+1} \Theta_{i_s}$ and $\bigcup_{k+1
      \leq s \leq 2k-l+1} \Theta_{i_s} \nsubseteq \bigcap_{r \leq k-l} \Theta_{i_r}$.
  \end{description}
  For $m = \max \{|I|: I \in \mathscr{I}\}$ above conditions need to
  hold for all $k,l$ $k \leq m-2$ and $\max\{0,2-k\} \leq l \leq
  k-1$.
\end{proof}


For notational purposes it is practical to define a dependence
relation ``$\sim$''.

\begin{definition}
  For $j \in S$ $I \subseteq S$ define $j \sim I$ iff there is a subset
  $J \subseteq I$ such that $\bigcap_J \Theta_i \subseteq \Theta_j$.
\end{definition}

It follows immediatly that $\nsim$ is independence in the above sense.

\begin{proposition}
  $I \in \mathscr{I}$ if and only if for all $i \in I$, $i \nsim
  I\setminus i$.
\end{proposition}

\begin{comment}
  Some interesting notes on above matroid.
  \begin{itemize}
  \item We have $\emptyset \sim i$ for all $i \in S$, but $i \nsim
    \emptyset$ unless $\Theta_i = \Theta$. Essentially this means that
    there are no loops in $M(\mathcal{H})$
  \item For hypotheses $H_i,H_j$ such that $\Theta_j \subset \Theta_i$
    we have $i \sim j$ and this for example is the case for combined
    non-inferiority superiority testing.
  \item The rank function of an index set $I$ is given by the
    cardinality of its maximally independent subsets, \ie $\rho I =
    \max \{|J|:J\subseteq I, J \in \mathscr{I}\}$.
  \item To give some intuition to condition \ref{cond:matroida} one
    may see that for the simple case of $J = \{i\}$ and $I = \{j,l\}$
    the condition requires that for parallels $i \sim j$ we have $i
    \nsim l$ as $i \sim l$ would imply also $j \sim l$ which is in
    contradiction to the independence of $I$. For $|J|>1$ the
    condition does not necesarily involve parallels.
  \item As a system of null hypotheses that does not conform to
    \ref{cond:matroida} consider the following set of hypotheses:
    \begin{align*}
      H_1:& (\mu_1 \leq 0) &, (\mu_2 \leq 0) \\
      H_2:& \mu_1 &\leq 0 \\
      H_3:& \mu_2 &\leq 0,
    \end{align*}
    according to Definition \ref{def:indep} $\{1\},\{2,3\} \in
    \mathscr{I}$, however $\Theta_1 = \Theta_2 \cap \Theta_3$. From a
    testing perspective we may simple drop $H_1$ as it is simply $H_2
    \cap H_3$ which would be tested anyways.
  \item Now consider for $\delta > 0$,
    \begin{align*}
      H_1:& (\mu_1 \leq 0) &, (\mu_2 \leq 0) \\
      H_2:& \mu_1 &\leq \delta \\
      H_3:& \mu_2 &\leq \delta.
    \end{align*}
    Again, Condition \ref{cond:matroida} does not hold. However, in
    this case replacing the system by an equivalent system of null
    hypotheses is not as straight forward. A solution may be to
    replace $H_1$ by:
    \begin{align*}
      H_{1a}:& \mu_1 &\leq 0 \\
      H_{1b}:& \mu_2 &\leq 0,
    \end{align*}
  \end{itemize}
\end{comment}

Next we show an important connection between matroid theory and
multiple testing theory

\begin{theorem}
  Consider a family of null hypotheses for which conditions
  (\ref{cond:matroida}-\ref{cond:matroidb}) hold. Then an index set $I \subseteq S$ is
  weakly exhaustive if and only if $I$ is a flat in $M$.
\end{theorem}

\begin{proof}
  Not yet done!
\end{proof}

To get a result on strictly exhaustive set we need to restrict
condition (\ref{cond:matroida}) further.

\begin{theorem}
  Consider a family of null hypotheses $\mathcal{H}$ for which above
  the more stringent conditions hold:
  \begin{align}
    \label{cond:matroid2}
    \bigcap_J \Theta_i &\nsubseteq \bigcup_{I\setminus J} \Theta_i,
    \text{ and} \\
    \bigcap_{J}\Theta_i &\nsupseteq     \bigcup_{I \setminus J} \Theta_i ,
  \end{align}
  % and additionally $\bigcap_S \Theta_i^C \neq
  % \emptyset$ (practically that means that all alternatives may be true
  % at the same time).
  Then independence as defined in Definition \ref{def:indep} gives a
  matroid for   which every flat corresponds to a strictly exhaustive
  index set and vice versa.
\end{theorem}

\begin{proof}
  Not yet done!
  % We first show by contradiction that every flat is strictly
  % exhaustive: Consider $I$ is a flat but that $I$ is not strictly
  % exhaustive. As $I$ is a flat, we have for every $l \in S\setminus
  % I$, $l \sim I$. Let $J \in \mathscr{I}$ be a maximally independent
  % set in $I$. Then $J \cup l \in \mathscr{I}$ and $|J \cup l| =
  % |J|+1$. If $I$ were not strictly exhaustive, then $\bigcap_{I}
  % \Theta_i \cap \bigcap_{S\setminus I} \Theta^c_i = \emptyset$. 
\end{proof}

\begin{comment}
  Above restriction (\ref{cond:matroid2}), already for the simple case
  of $|J|=1$, is no longer a statment about the parallels of
  $M$. 
\end{comment}



On the other hand if we look at a system of hypotheses and consider
exhaustiveness as defined in [hommel, bergmann] we can define a
matroid. 

\begin{theorem}
\label{the:matroid}
  Consider a (finite) family of null hypotheses $\mathcal{H} = \{H_i,
  i \in S\}$ with assiciated index set $S$. Let $\mathcal{P}(S) = \{I
  \subseteq S\}$ denote the power set of $S$ and $SE:\mathcal{P}(S)
  \rightarrow \mathcal{P}(S)$ the function that assigns to each $I
  \subseteq S$ the smallest exhaustive set $SE(I)$ such that $I \in
  SE(I)$. Taken as a closure operator $SE$ defines a matroid $M$ on
  $\mathcal{P(S)}$.
\end{theorem}

\begin{proof}
  We need to show that the closure axioms hold for $SE$:
  \begin{enumerate}
  \item $I \subseteq SE(I)$ holds by definition
  \item $J \subseteq I \Rightarrow SE(J) \subseteq SE(I)$ and,
  \item $SE(I) = SE(SE(I))$ and,
  \item $j \notin I$, $j \in SE(I \cup i) \rightarrow i \in SE(I \cup
    j)$ follow immediately from Lemma $1.5$ in [Bernhard, 91].
  \end{enumerate}
\end{proof}

\subsection{Matroid of tPCP}

\begin{definition}
  Let $\dot{S}$ be a tPCP with associated comparigram $G(\dot{S})$ we
  denote the cycle matroid of $G(\dot{S})$ by $M(\dot{S})$ or simply
  $M$ in case were clear from the context. 
\end{definition}

For convenience we reproduce here some of the properties of the cycle
matroid $M$ of a graph $G$. 

\begin{proposition}{Properties of $M$ (Welsh, 2010)}
\label{the:cyclemat}
  \begin{enumerate}
  \item The cycles of $G$ are the circuits of $M$ 
  \item The spanning forests of $G$ are the bases of $M$
  \item If $G$ is connected the bases of $M$ are the spanning trees of $G$
  \item A set $X$ of edges of $G$ is independent in $M$ if and only if
    $X$ contains no cycle
  \item The rank of $M$ is $|V(G)| - k(G)$ where $k(G)$ denotes the
    number of connected components in $G$
  \item For any subset $A \subseteq E(G)$ the rank of $A$ in $M$ is
    given by
    \begin{displaymath}
      \rho_G(A) = |V(A)| - k(A)
    \end{displaymath}
    where $V(A)$ and $k(A)$ denote the set of vertices and number of
    connected componenents in the subgraph of $G$ induced by $A$
  \item Loops, and parallel edges correspond in $G$ and $M$
  \item Let $A \subset E(G)$, let $e \in E(G)\setminus A$. Then $e$
    belongs to the closure of $A$ in $M(G)$ if and only if there is a
    cycle $C$ of $G$ with
    \begin{displaymath}
      e \in C \subseteq A \cup e.
    \end{displaymath}

  \end{enumerate}
\end{proposition}

Next we give some useful notation and results on exhaustive index sets
of pairwise comparisons.

\begin{proposition}
  Let $X \subseteq S$ then there is a unique collection of mutually
  exclusive index sets $s(X) = \{X_i = \{i_1,...,i_{m_i}\} \subseteq
  V(\dot{S}), i \in \{1,...,M\}: X_i \cap X_j = \emptyset\}$, such
  that $H_X$ is the intersection of hypotheses of the form $H_{X_i}:
  \mu_{i_1} = ... = \mu_{i_{m_i}}$, \ie $H_X = \bigcap_{1:M}
  H_{X_i}$. We call $s(X)$ the {\em associated partition hypothesis}.
\end{proposition}

It follows directly that 

\begin{theorem}
  Consider $I, J \in R$, $H_I = H_J$ if and only if $s(I) = s(J)$. 
\end{theorem}

The following Lemma will be helpfull for the proof of Theorem
\ref{the:compmat}. 

\begin{lemma}
  Let $X \subseteq R$ then $H_{s(X)} = H_X$. 
\end{lemma}

\begin{proof}
  Consider that $H_{\sigma X} \neq H_X$. $\sigma X$ is exhaustive and
  $X \subseteq \sigma X$ follows by definition. Therefore, $H_{\sigma
    X}$ is a proper subset of $H_X$. Consequently there exist $A \in
  s(X)$ and $B \in s(\sigma X)$ such that $A$ is a proper subset of
  $B$. Now consider $\bigcup_{X_i \in (s(\sigma X)\setminus B)\cup A}
  H_{X_i}$ and $Y = \{(a,b) \in R: a,b \in X_i \in (s(\sigma
  X)\setminus B)\cup A\}$ then $X \subset Y$ and since $H_x \subset
  H_Y$ and $X$ exhaustive $Y \subset \sigma X$. However, $Y$ is
  exhaustive, which contradicts that $\sigma X$ is the smallest
  exhaustive index set in $\mathcal{P}(R)$.
\end{proof}

From this one immediately sees how the smallest containing exhaustive
index sets can be constructed.

\begin{corollary}
  $SE(X) = \{(a,b) \in R:\exists A \in s(X), a,b \in A\}$
\end{corollary}


Using these properties we show the following theorem. 

\begin{theorem}
  Consider a tPCP $\dot{S}$ with comparigram $G$. Let $M$ be the
  associated cycle matroid of $G$ then an index set $I \subseteq R$ is
  exhaustive if and only if it is a flat in $M$.
\end{theorem}

\begin{proof}
  Consider $I$ exhaustive and assume that is not a flat. Let $x \in
  \sigma I \setminus I$. Then $\rho(I) = |V(I)| - k(I) = \rho(I \cup
  x) = |V(I \cup x)| - k(I \cup x)$. By adding an edge to $I$, $|V(I
  \cup x)| = |V(I)| + d$ where $d$ is either $0$, $1$ or $2$. Iince
  $G$ simple and $I$ exhaustive, any endpoints of $x$ in $V(I)$ has to
  be isolated in $I$. For $d = 0$, $x$ connects isolated vertices $a,
  b \in V(I)$ accordingly that $k(I \cup x) = k(I) - 1$ which is a
  contradiction. Iimilarly, $d = 1$ gives $k(I\cup x) = k(I)$ and $d =
  2$ leads to $k(I\cup x) = k(I)+1$ in either case the rank of $I$ is
  increased by $1$.

  Assume that $I$ is a flat in $M$ but $I$ is not
  exhaustive. Consequently there exists $I' \subseteq R$ such that $I$
  is a proper subset of $I'$ and $H_I = H_{I'}$. Therefore we have
  $s(I) = s(I')$. Let $e = (a,b) \in  I'\setminus I$ then there is $A
  \in s(I)$ such that $a,b \in A$. As $e \notin I$, there is at least
  one $c \in A$ such that $(a,c),(b,c) \in I$. Now
  $C = \{(a,b),(b,c),(a,c)\} \in \mathcal{C}$ such that $(a,b) \in
  C\setminus I$ which contradicts that $I$ is flat.
\end{proof}

We next show how the operation that for each index set identifies the
smallest containing exhaustive index set induces a matroid.


\begin{theorem}
  Consider a tPCP $\dot{S}$ for any $I \subseteq R$ let
  \begin{displaymath}
    \hat{\sigma} I = SE(I)
  \end{displaymath}
  where $SE(I)$ is the smallest exhaustive index set in
  $\mathcal{P}(R)$ cointaining $I$. Then $\hat{\sigma} I$ is the closure
  operator of a matroid $M$.
\label{the:compmat}
\end{theorem}


We now proof Theorem \ref{the:compmat}.

\begin{proof}
  Actually this follows directly from Theorem \ref{the:matroid}. I
  leave the old proof for completeness.

  We need to show
  \begin{enumerate}
  \item $X \subseteq \hat{\sigma} X$, 
  \item For $Y \subseteq X$ then $\hat{\sigma}(Y) \subseteq \hat{\sigma}(X)$, and 
  \item $\hat{\sigma} X = \hat{\sigma} \hat{\sigma} X$
  \item If $y \notin \hat{\sigma} X$, $y \in \hat{\sigma}(X \cup x)$ then $x \in
    \hat{\sigma}(X \cup y)$.
  \end{enumerate}
  (1) holds by definition. 

  (2) assume $Y \subset X$ and $\hat{\sigma} Y \setminus \hat{\sigma} X \neq
  \emptyset$. Let $h \in \hat{\sigma} Y \setminus \hat{\sigma} X$ and let $h =
  (a,b)$. Then there is $A \in s(Y)$ with $a,b \in A$ but no $B \in
  s(X)$ with $a,b \in B$. Since $Y \subset X$, for all $A \in s(Y)$,
  $B \in s(X)$ such that $A \subset B$. Which leads to a
  contradiction. 

  (3) follows trivially. 

  (4) let $x = (a,b)$, there is no $A \in s(X)$ with $a,b \in A$ but
  there is some $B \in s(X \cup y)$ with $a,b \in B$. Let $y = (c,d)$
  we can distinguish 3 cases: (a) $\{c,d\} \cap B = \emptyset$ for all
  $B \in s(X)$ then $s(X \cup y) = s(X) \cup \{c,d\}$ such that $B =
  \{c,d\}$ and $x = y$. (b) Ther is $C \in s(X)$, $c \in C$ but
  no $D \in s(X)$ with $d \in D$. Then $s(X \cup y) = (s(X)\setminus C) \cup (C
  \cup d)$ and since this is the only set that has changed by adding
  $y$, $x \in (C\cup d)$, therefore $b = d$ and consequently
  $c,d \in C\cup b$ and also $y \in \hat{\sigma}(X \cup x)$. (c) There are
  $C,D \in s(X)$ such that $c \in C$ and $d \in D$ then $s(X \cup y) =
  (s(X)\setminus \{C,D\})\cup (C \cup D)$ and since $C\cup D$ is the
  only set changed from $s(X)$ $a \in C$ and $b \in D$ such that also
  $y \in \hat{\sigma}(X\cup x)$. 
\end{proof}

As the cycle matroid of the comparigram and the matroid defined by
$\hat{\sigma}$ have identical closed sets it follows directly that
they are the same matroid

\begin{corollary}
  The matroid $\hat{M}$ on $\dot{S}$ defined by the closure operation 
  $\hat{\sigma}I = SE(I)$, $I \subseteq \dot{S}$ is the cycle matroid
  of the comparigram $G(\dot{S})$
\end{corollary}

Accordingly all porperties given in \ref{the:cyclemat} also hold for
the matroid on exhaustive sets. 

The following theorem shows an important connection between the flats
of the cycle matroid of a complete graph and the partitions of a
finite set. 

\begin{theorem}{[Welsh]}
\label{the:partition.lattice}
  The lattice of flats of $M(K_n)$ is isomorphic to the distributive
  lattice of all partitions of a set with $n$ elements ordered by
  refinement. 
\end{theorem}

Translated to pairwise comparison problems this means that the system
of exhaustive sets and the associated partial order induced by the 
inclusion operator is just corresponds to the partitions of the set of
independent population means ordered by refinement. This is not
surprising as $s(X)$ associates each index set with a partition of the
set $V(R)$ and restricted to the family of flats $\mathcal{L}(R)$,
$s(X)$ is one to one. Recognizing that any partition of means
corresponds to a decomposition of $K_n$ into cliques this provides an
alternative proof to the main result of Weichert on two-sided pairwise
comparisons. 

The properties of set partitions have been already extensively
studied. For example, this correspondence provides us with efficient
algorithms to generate all exhaustive sets. Further, the number of
exhaustive sets resulting from all pairwise compinations is given by
the Bell numbers.

If not all pairwise comparisons are of interest we can still apply
Theorem \ref{the:partition.lattice} using the following results on
matroid minors. Let $|$ and $.$ denote the operators for restriction
and contraction, then

\begin{theorem}{Theorem 2 [Welsh, p.66]}
  If $\mathcal{L}$ is the geometric lattice of the simple matroid M,
  the interval $[o,T]$ is the lattice of $M|T$, for any flat $T$ of
  $M$. 
\end{theorem}

\begin{theorem}
\label{the:interval}
  Consider a tPCP $\dot{S} \in \dot{\mathcal{T}}(n)$. The exhaustive
  sets of $\dot{S}$ generate a geometric lattice
  $\mathcal{L}(\dot{S})$ that is isomorphic to: 
  \begin{displaymath}
    [\emptyset,\sigma_{K_n} S] \cap S,
  \end{displaymath}
  which is the interval sublatticee of the partition lattice of rank
  $n$ where for each subset the intersected with the set of
  comparisons $S$ is taken. 
\end{theorem}

\begin{proof}
  The closure operator of $M|S$ is given by $\sigma_R(A) = \sigma A
  \cap S$. Consider $L = \mathcal{L}(M|S)$ and $K = \mathcal{L}(M|\sigma
  S)$ then it can be easily seen that for $A,B \in K$ there are unique $A',B'
  \in K \cap R$ such that $(A \land B)' = A' \land B'$ and $(A
  \lor B)' = A' \lor B'$. 
\end{proof}

Above result does show a way how to construct the exhaustive sets for
a general pairwise comparison problem. The intervall lattice
$[\emptyset,=\sigma_{K_n} S]$ can be constrcuted by refinement of
$V(\sigma_{K_n} R)$.  However, if $\sigma_{K_n} S = K_n$ the
corresponding intervall lattice is identical to the full
lattice. Nevertheless one can reduce the problem in the following
way. Perfom one refinement step of $\sigma_{K_n} S$ and intersect with
$S$ to get all hyperplanes of $M|S$. Then all consecutive flats of
$M|S$ can be found by closing the set of hyperplanes with regard to
intersection.

\subsection{Hierarchical testing of pairwise comparisons}

\begin{theorem}
  Consider a hierarchical testing procedure that tests $H_{i_r}$ at
  level $\alpha$ according to a prefixed order $(i_1,...,i_n)$. Assume
  that hypotheses with indeces $i_1$ through $i_k$ have been
  rejected. If for $k+1$ to $k+l$ there is $i_s$, $s \leq k$ such that
  \begin{equation}
    \label{eq:condh}
    \rho \{i_s,i_{k+1}\} = \rho \{i_s,i_{k+1}...,i_{k+l}\},
  \end{equation}
  hypotheses each $H_{i_{k+1}}$ through   $H_{i_{k+l}}$ can be tested
  simultaneously at full level $\alpha$. 
\end{theorem}

\begin{proof}
  Due to \ref{eq:condh} for any $k+1 \leq m' < m \leq k+l$, $i_m \in
  \sigma \{i_s,i_{m'}\}$ and consequently also $i_s \in \sigma
  \{i_m,i_{m'}\}$. Therefore $i_s \in \sigma J$ for all $J$ with
  $i_m,i_{m'} \in J$. Since $p_s \leq \alpha$ all intersection
  hypotheses corresponding to exhaustive sets containing $i_{m}$
  and $i_{m'}$ for all $k+1 \leq m' < m$ have been rejected. Such that
  $p_{i_{m}} \leq \alpha$ rejects $H_{i_{m}}$.
\end{proof}

\subsection{Graphical procedure for pairwise comparison problems}

Above properties of tPCPs can be used to derive weighted multiple
testing procedures, that
\begin{enumerate}
\item control the FWER,
\item use a weighting strategy defined by a graph,
\item exploits logical dependencies between hypotheses to yield
  improved power,
\item produces significance statements that are consistent with
  logical dependencies between hypotheses.
\end{enumerate}

Some additional notation let $\gmcp$ denote a graphical weighting
procedure, such that $\gmcp(I)$ gives the weights for intersection
hypothesis $H = \bigcup_{i \in I} H_i$. $gMCP(n)$ is the class of all
graphical procedures for $n$ hypotheses. Define $h(S)$ the set of
hyperplanes of $S$ in $M$. 

\begin{algorithmic}
  \REQUIRE $\dot{S} \in \dot{\mathcal{T}(m)}$, $\gmcp \in gMCP(n)$, 
  $p \in [0,1]^n$, $\alpha$
  \ENSURE $n = |E(\dot{S})| \leq m*(m-1)/2, \alpha \in [0,1]$
  \STATE $w \gets \gmcp(R)$
  \STATE $R \gets \{i \in R: p_i \leq w_i \alpha\}$
  \STATE $S \gets S \setminus R$
  \STATE $EI \gets h(\dot(S))$
  \WHILE{$|R| < n$}
  \STATE For all $E \in EI$, $w_E \gets \gmcp(E)$
  \STATE For all $i \in S\setminus R$, $w_i^\star \gets \min_{E \in
    EI}w_{i,E}$
  \STATE $r \gets \{i:p_i \leq w_i^\star*\alpha$
  \STATE $v \gets \{i \in S\setminus R:i \notin E, \forall E \in EI\}$
  \IF{$r = v = \emptyset$}
  \STATE Reject $R$ and stop
  \ENDIF
  \STATE $R \gets R \cup r$
  \STATE $S \gets S\setminus r$
  \STATE $EI \gets h(\dot(S))$
  \ENDWHILE
  \STATE Reject $R$ and stop
\end{algorithmic}

Taking the minimum weight accross all flats in above procedure may be
improved:

\begin{algorithmic}
  \REQUIRE $\dot{S} \in \dot{\mathcal{T}(m)}$, $\gmcp \in gMCP(n)$, 
  $p \in [0,1]^n$, $\alpha$
  \ENSURE $n = |E(\dot{S})| \leq m*(m-1)/2, \alpha \in [0,1]$
  \STATE $w \gets \gmcp(R)$
  \STATE $R \gets \{i \in R: p_i \leq w_i \alpha\}$
  \STATE $S \gets S \setminus R$
  \STATE $EI \gets h(\dot(S))$
  \WHILE{$|R| < n$}
  \STATE For all $E \in EI$, $w_E \gets \gmcp(E)$
  \STATE $r \gets \{i:\forall E \in EI, i \in E,p_i \leq w_i \alpha\}$
  \STATE $v \gets \{i \in S\setminus R:i \notin E, \forall E \in EI\}$
  \IF{$r = v = \emptyset$}
  \STATE Reject $R$ and stop
  \ENDIF
  \STATE $R \gets R \cup r$
  \STATE $S \gets S\setminus r$
  \STATE $EI \gets h(\dot(S))$
  \ENDWHILE
  \STATE Reject $R$ and stop
\end{algorithmic}

Comments: Above procedures are valid for all families of hypotheses
for which exhaustive sets are identical with strictly exhaustive
sets. The crucial part of the algorithms is finding $h(S)$. For $tPCP$
this can be achieved using Theorem \ref{the:interval}.





\subsection{General contrast tests}

Realizing that each pairwise comparison may be represented by a
general contrast test we next investigate the connections between such
tests and vectorial matroids. 

Some interesting observations:
\begin{itemize}
\item Pairwise comparisons (as a contrast) are represented by a vector
  having a $1$ at the component corresponding to the first and a $-1$
  at the component corresponding to the second mean in the
  comparison. For example if $m = 4$, $H_{13}: \mu_1 = \mu_3$
  corresponds to  $c_{13} = (1,0,-1,0)$ 
\item If we consider only two-sided hypotheses the $c_{ij} = -c_{ji}$ and
  $-c_{ij} = c_{ji}$  represent the same hypothesis. Consequently,
  without loss of generality we can represent $H_{ij}$ by a vector
  $e_{ij}$ over $GF(2)$ with entries $1$ at $i$ and $j$.
\item From representation theory over matroids we now that every graph,
  and consequently tPCP, may be represented over $GF(2)$ by using the
  incidence matrix.
\item Given that a representation is defined as a map that conserves
  the rank. The maximum rank of a tPCP is $m-1$.
\item It can be easily seen that the corresponding contrast matrix
  maybe reduced to a rank $\leq m-1$ upper diagonal matrix.
\item A general contrast test may, however, have up to rank $m$ and
  can not be represented over $GF(2)$.
\end{itemize}

\begin{definition}
  Consider $X \sim N(\bs{\mu},\sigma \mathbb{1}_m)$ and a contrast $A
  \in \mathbb{R}^{m \times n}$ with column vectors $a_i \in
  \mathcal{R}^m$. We call the MTP given by by the family $\mathcal{H}
  = \left\{H_i: a_i\mu = 0; i \in \{1,...,n\}\right\}$ a (finite)
  linear  contrast problem (LCP).
\end{definition}


\section{One sided problems}

So far we have only considered two sided hypotheses \ie point
hypotheses. In many practical applications however the questions of
interest are better addressed by one sided null hypotheses. For
example in a clinical trial one generally only interested whether the
a new treatment is better than some comparator. 

From conditions \ref{} it can be easily seen that while problems of
one sided hypotheses may be treated by matroids, already for simple
problems the correspondence between flats and strictly exhaustive sets
is not one to one. Nevertheless an alternative approach to this class
of problems can be found in the theory of oriented matroids.

First let us define a one sided multiple testing problem

\begin{definition}
  Let $\bs{X}$ denote an $m$ dimensional random vector with
  distribution function $P_{\bs{\mu}}$ from some parametric family
  $\mathscr{P} = \{P_{\bs{\theta}}: \bs{\theta} \in \mathbb{R}^m\}$. A
  family of {\em one sided linear contrast hypotheses} is defined by a
  contrast matrix $\bs{A} \in \mathbb{R}^{m\times n}$ and vector
  $\bs{b} \in \mathbb{R}^n$ that define a the family of null
  hypotheses of the form $\mathscr{H} = \{H_i: \bs{A_i}^t\bs{\mu} \leq
  b_i,i \in \{1,...,n\}\}$, where $\bs{A}_i$ denotes the $i$-th column
  of $\bs{A}$ and $b_i$ the $i$-th componenth of $\bs{b}$.
\end{definition}

Now observe that each hypothesis $H_i$ corresponds to a hyperplane
$h_i = \{\bs{x} \in \mathbb{R}^m: \bs{A}_i^t\bs{x} = b_i\}$ that divides
$\mathbb{R}^m$ into two regions $h_i^+ = \{\bs{x} \in \mathbb{R}^m:
\bs{A}_i^t\bs{x} < b_i\}$ and $h_i^- = \{\bs{x} \in \mathbb{R}^m: \bs{A}_i^t\bs{x} >
b_i\}$ giving rise to a sign function:
\begin{equation}
  \label{eq:sign}
  \delta_i(\bs{x}) = \left\{
\begin{array}{ll}
  + & \text{if } \bs{x} \in h_i^+\\
  0 & \text{if } \bs{x} \in h_i \\
  - & \text{if } \bs{x} \in h_i^-\\
\end{array}\right.
\end{equation}

Further for every $\bs{x} \in \mathbb{R}^m$ we have  sign vector
$\bs{\delta}{\bs{x}}= (\delta_1(\bs{x}),...,\delta_n(\bs{x}))$. Now
for each $i$ and some parameter value $\bs{\theta} \in \bs{\Theta} = \R^m$,
$\delta_i(\bs{\theta})$ tells us whether the corresponding value is under the null
hypotheses on its border, or under the alternative. Accordingly,
$\bs{\delta}(\bs{\theta})$ tells us in which combination of true and
false null hypotheses $H_i$ correspond to $\bs{\theta}$. Let
\begin{equation}
\label{eq:faces}
\mathcal{F}(\bs{A},\bs{b} = \{\bs{\delta}(\bs{x}): \bs{x} \in \R^m\}
\end{equation}

and for some $F \in \mathcal{F}$ define
\begin{equation}
  \label{eq:partition}
  \Theta_F = \{\bs{x} \in \R^m: \bs{\delta}(\bs{x}) = F\}
\end{equation}

then we have $\bigcup_{F \in \mathcal{F}} \Theta_F = \bs{\Theta} =
\R^m$ and for any $\Theta_F \cap \Theta_G = \emptyset$ for any $F \neq
G$. Consequently, $\{\Theta_F:F \in \mathcal{F}$ represents a
partition of the parameter space and each element in $\mathcal{F}$
uniquely identifies some {\em non-empty} element of that partition. Consequently,
following the Partitioning Principle \cite[]{finner2002partitioning}
if we have level $\alpha$ tests $\varphi_F(X)$ for each $F \in
\mathcal{F}$ a test that rejects $H_i$ for observed values $X$ if
$\varphi_F(X) = 1$ for all $F$ with $F_i \geq 1$ controls the FWER in
the strong sense. For example one may use  weighted Bonferroni tests
\begin{equation}
  \label{eq:bonferroni}
  \varphi_F = \indf{\min_{i:F_i \geq 0}p_i \leq w_{F,i}}
\end{equation}
where the $w_{F,i}$ are can be computed using a graphical approach.

Note that as $\Theta_F$, $F \in \mathcal{F}$ the $\mathscr{L} = \{I \subset \{1,...,m\}: \exists F \in
\mathcal{F}: F_i \geq 0 \forall i \in I\}$ are all strictly exhaustive sets.  

The question remains how to compute $\mathcal{F}$. For this observe
that the $h_i$ define a hyperplane configuration and $\mathcal{F}$ is
just the face lattice of the corresponding oriented
matroid. Consequently, $\mathcal{F}$ can be found using algorithms
for arrangement construction from polyhedral programming
\cite{Avis-KF-92,Avis-KF-96,Ferrez-KF-Liebling-01} which are
polynomial in time and input size and compact in space!


\begin{definition}
  Consider a family of hypotheses $\mathscr{H} =$
\end{definition}



\end{document}
